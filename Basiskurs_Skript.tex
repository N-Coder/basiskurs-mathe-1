\documentclass[11pt,a4paper,ngerman]{article}

% IMPORTS
\usepackage[utf8]{inputenc}
%\usepackage[ngerman]{babel}
\usepackage{amsmath}
\usepackage{relsize} %Größere Symbole
\usepackage{amsfonts} % Buchstaben mit Doppelstrich
\usepackage{parskip} % Absätze ohne Einrückungen
\usepackage{mathtools} % Für Pfeil mit Erklärung
\usepackage[colorinlistoftodos,prependcaption,textsize=tiny]{todonotes} % To-Dos
\usepackage{url}
\usepackage{hyperref}

% PFEIL MIT ERKLÄRUNG
\newcommand{\equalexpl}[1]{%
  \underset{\substack{\uparrow\\\mathrlap{\text{\hspace{-1em}#1}}}}{\equiv}}
\newcommand{\circexpl}[1]{%
  \underset{\substack{\uparrow\\\mathrlap{\text{\hspace{-1em}#1}}}}{\circ}}

\newcommand{\mathlll}[1]{\mathlarger{\mathlarger{\mathlarger{#1}}}}
\newcommand{\mathll}[1]{\mathlarger{\mathlarger{{#1}}}}

% META
\date{\today}
\author{Maximilian Reif}
\title{Basiskurs Mathematik}

%--------------------------------------------------------------------------------------------------------------

\begin{document}

\begin{titlepage}
    \ \newline\newline\newline\newline\newline
	\begin{center}
		\huge Mitschrift zum \\
		\Huge \textbf{Basiskurs Mathematik} \\
		\huge bei Prof. Kreuzer im WS 16/17 \\
		\normalsize

		\vspace{1cm}
		\begin{tabular}[b]{l|l}
			\textbf{author} & Maximilian Reif \texttt{\href{mailto:reifmaxi@fim.uni-passau.de}
			{<reifmaxi@fim.uni-passau.de>}} \\
			\textbf{last change} & \today, version 0.8.0 \\
			\textbf{github} & \url{https://github.com/lordreif/basiskurs-mathe}
		\end{tabular}

		\vspace{1cm}
		
	\end{center}
   
	
\end{titlepage}

\setcounter{page}{2}

\newpage
\tableofcontents \thispagestyle{empty}
\newpage

\section{Rechnen mit ganzen Zahlen}

$\mathbb{N} = \{0, 1, 2, 3,  \dots \}$ Menge der \textbf{natürlichen Zahlen}\\
$\mathbb{Z} = \{0, 1, -1, 2, -2,  \dots \}$ Menge der \textbf{ganzen Zahlen}


\subsection[Zahlensysteme]{Satz (Zahlensysteme)}

	Sei $b\in\mathbb{N}\text{ mit }b\geq 2$. 
	(Die Zahl $b$ heißt \textbf{Basis} des Zahlensystems)\\
	Dann gibt es zu jeder natürlichen Zahl $n\in\mathbb{N}$ eindeutig bestimmte Elemente 
	$a_0,a_1,\dots,a_k\in\{0,1,\dots,b-1\}$ sodass gilt:

	\[
		n = a_0 + a_1 \cdot b + a_2 \cdot b^2 + \cdots + a_k \cdot b^k.
	\]

	Die Zahlen $a_0,\dots,a_k$ heißen \textbf{Ziffern} von $n$ 
	in der Darstellung zur Basis $b$.\\
	\underline{Schreibweise:} $n_{[b]}=a_k a_{k-1}\dots a_1 a_0$ 
	(fehlt $[b]$ so ist $[10]$ gemeint)


\subsection{Beispiele}

	\begin{itemize}
	
		\item Binärsystem, $b=2$

		\begin{itemize}
		
			\item $5_{[10]}=101_{[2]}$
			\item $101_{[10]}=64_{[10]}+32_{[10]}+4_{[10]}+1_{[10]}=1100101_{[2]}$
		
		\end{itemize}
		
		\item Hexadezimalsystem, $b=16$\\
		\underline{Notation:} 
		$10_{[10]}=A_{[16]},11_{[10]}=B_{[16]},\dots,15_{[10]}=F_{[16]}$
		
		\begin{itemize}
		
			\item $101_{[10]}=5\cdot 16+5=55_{[16]}$
			\item $1B3_{[16]}=256_{[10]}+11_{[10]}\cdot 16_{[10]}+3_{[10]}=435_{[10]}$
			
		\end{itemize}
	
	\end{itemize}
	

\subsection[Division mit Rest]{Satz (Division mit Rest)}

	Sei $n\in\mathbb{Z}$ und $m\in\mathbb{N_+}$.\\
	Dann gibt es eine eindeutige Darstellung
	$n=q\cdot m+r$ mit $q\in \mathbb{Z}$ (genannt \textbf{Quotient})
	und $r\in\{0,1,\dots,m-1\}$ (genannt \textbf{Rest}).
	
	\[	
		\text{\underline{Schreibweise:} }n\equalexpl{"ist kongruent"}r
		\qquad(\operatorname{mod}~m)		
	\]

	
\subsection{Beispiele}
	
	\begin{itemize}

		\item Die Zahl $n=87$ soll durch $m=5$ geteilt werden:
		
		\[
			n=q\cdot m+r=17\cdot 5+2
		\]			
		
		\item Die möglichen Reste bei der Division einer Quadratzahl durch 12 sind:\\
		
		\begin{tabular}{r|c|c|c|c|c|c|c|c|c|c|c|c|}
		$n$ \ (mod 12)		& 0 & 1 & 2 & 3 & 4 & 5 & 6 & 7 & 8 & 9 & 10 & 11 \\ \hline
		$n^2$ \ (mod 12)	& 0 & 1 & 4 & 9 & 4 & 1 & 0 & 1 & 4 & 9 & 4  & 1
		\end{tabular}
	
	\end{itemize}

	
\subsection[Vielfaches, Teiler, Primzahl]{Definition (Vielfaches, Teiler, Primzahl)}

	\begin{enumerate}
		
		\item Ist der Rest bei der Division von $n$ durch $m$ gleich Null, 
		so heißt $n$ ein \textbf{Vielfaches} von $m$ und $m$ ein \textbf{Teiler} von $n$.
		
		\item Eine Zahl $n\geq 2$ heißt eine \textbf{Primzahl}, wenn sie nur 
		zwei positive Teiler 1 und $n$ besitzt.
		
	\end{enumerate}


\subsection{Beispiele}

	\begin{itemize}
		
		\item Die Teiler von 12 sind 1, 2, 3, 4, 6, 12.
		\item Die ersten Primzahlen sind 2, 3, 5, 7, 11, 13, 17, 19, \dots
		
	\end{itemize}
	
	
\subsection[Fundamentalsatz der Arithmetik]{Satz (Fundamentalsatz der Arithmetik)}

	Sei $n\in\mathbb{N_+}$. Dann gibt es eine (bis auf die Reihenfolge)
	eindeutige Darstellung
	
	\[
		n=p_1^{\alpha_1}\cdot p_2^{\alpha_2}\cdot\ldots\cdot p_n^{\alpha_n}
	\]
	
	mit paarweise verschiedenen Primzahlen $p_1,\dots,p_k$ und $\alpha_i\in\mathbb{N_+}$.\\
	Diese Darstellung heißt \textbf{Primfaktorzerlegung} von $n$.

	
\subsection{Beispiele}

	\begin{itemize}
	
		\item $24=2^3\cdot 3$
		\item $111=3\cdot 37$
		\item $1011=7\cdot 11\cdot 13$
		\item $1024=2^{10}$
		\item $729=3^6$
		\item $625=5^4$
	
	\end{itemize}
	
	
\subsection[$\operatorname{ggT}, \operatorname{kgV}$]{Definition ($\operatorname{ggT}, \operatorname{kgV}$)}
	
	Seien $a,b\in\mathbb{N_+}$.
	
	\begin{enumerate}
	
		\item Die größte positive ganze Zahl $g\in\mathbb{N_+}$ mit $g|a$ und $g|b$
		heißt der \textbf{größte gemeinsame Teiler ($\operatorname{ggT}$)} von $a$ und $b$.
		
		\item Die kleinste positive ganze Zahl $k\in\mathbb{N_+}$ mit $a|k$ und $b|k$
		heißt das \textbf{kleinste gemeinsame Vielfache ($\operatorname{kgV}$)}
		von $a$ und $b$.
		
	\end{enumerate}


\subsection[$\operatorname{ggT}$/$\operatorname{kgV}$ durch Primfaktorenzerlegung]{Satz ($\operatorname{ggT}$/$\operatorname{kgV}$ durch Primfaktorenzerlegung)}

	Sei $a,b\in\mathbb{N_+}$ mit Primfaktorzerlegungen 

	\[
		a=p_1^{\alpha_1}\cdot\ldots\cdot p_k^{\alpha_k}\text{ und}
		b=p_1^{\beta_1}\cdot\ldots\cdot p_k^{\beta_k}\text{ mit }\alpha_i,\beta_i\geq 0.
	\]
	
	Dann gilt:
	
	\begin{enumerate}
		
		\item $\operatorname{ggT}(a,b)=p_1^{\gamma_1}\cdot p_2^{\gamma_2}
		\cdot\ldots\cdot p_k^{\gamma_k}$ mit $\gamma_i=\operatorname{min}
		\{\alpha_i,\beta_i\}$
		
		\item $\operatorname{kgV}(a,b)=p_1^{\delta1}\cdot p_2^{\delta2}
		\cdot\ldots\cdot p_k^{\delta_k}$ mit $\delta_i=\oparatorname{max}
		\{\alpha_i,\beta_i\}$
		
	\end{enumerate}
	
		
\subsection{Beispiele}

	\begin{itemize}
			
	\item $\operatorname{ggT}(30,75)=2^0\cdot 3^1\cdot 5^1 = 15$,\\ 
	denn $30=2\cdot 3\cdot 5$ und $75=3\cdot 5^2$
	
	\item $\operatorname{ggT}(64,81)=1$,\\ denn $64=2^6,81=3^4$

	\end{itemize}	

	
\subsection[Teilbarkeitsregeln]{Bemerkung (Teilbarkeitsregeln)}	

	\begin{enumerate}
	
		\item $2|n$ genau dann, wenn die Endziffer von $n$ in $\{0,2,4,6,8\}$ ist.
		\item $3|n$ genau dann, wenn die Quersumme($\operatorname{Qs}$)
		von $n$ durch 3 Teilbar ist.
		\item $4|n$ genau dann, wenn $4|(10a_1+a_0)$.
		\item $5|n$ genau dann, wenn $a_0\in\{0,5\}$ gilt.
		\item $6|n$ genau dann, wenn $2|n$ und $3|n$.
		\item $8|n$ genau dann, wenn $8|(100a_2+10a_1+a_0)$.
		\item $9|n$ genau dann, wenn $9|\operatorname{Qs}(n)$.
		\item $10|n$ genau dann, wenn $a_0=0$ gilt.
		\item $11|n$ genau dann, wenn $11|(a_0-a_1+a_2-+\cdots\pm a_k)$.
		\item $12|n$ genau dann, wenn $3|n$ und $4|n$.
	
	\end{enumerate}


\subsection{Beispiele}

	 \begin{itemize}
	 
	 	\item $9|123453$
	 	\item $11|1232$
	 
	 \end{itemize}


\subsection[Geschicktes Rechnen]{Bemerkung (Geschicktes Rechnen)}

	\begin{enumerate}
	
		\item Dritte binomische Formel: $(x-y)(x+y)=x^2-y^2$ plus Quadratzahlen
		
		\begin{itemize}
		
			\item $13\cdot 17=15^2-2^2=225-4=221$
			\item $23\cdot 25=576-1=575$
			\item $27\cdot 33=900-9=891$ 
			
		\end{itemize}		
		
		\item Multiplikation durch Umsortierung der Primfaktoren
			
		\begin{itemize}
	
			\item $8\cdot 375=8\cdot 3\cdot 125=10^3\cdot 3=3000$
			\item $40\cdot 75=4\cdot 10\cdot 3\cdot 25=3000$
			
		\end{itemize}
		
		\item Quadrieren mittels erster binomischer Formel: $(x+y)^2=x^2+2xy+y^2$
		
		\begin{itemize}
		
			\item $43^2=40^2+2\cdot 3\cdot 40+9=1600+240+9=1849$
			\item $98^2\cdot (100-2)^2=10000-400+4=9604$
			
		\end{itemize}

	\end{enumerate}\todo{Querverweise binom}


\subsection[Rekursive Definition von $\operatorname{ggT}$ und $\operatorname{kgV}$]{Definition (Rekursive Definition von $\operatorname{ggT}$ und $\operatorname{kgV}$)}

	Für $n\geq 2$ und $a_0,\dots,a_n\in\mathbb{N_+}$ gilt:
	
	\begin{itemize}
		
		\item $\operatorname{ggT}(a_1,a_2,\dots,a_n)=
		\operatorname{ggT}(\operatorname{ggT}(a_1,a_2,\dots,a_{n-1}),a_n)$
		\item $\operatorname{kgV}(a_1,a_2,\dots,a_n)=
		\operatorname{kgV}(\operatorname{kgV}(a_1,a_2,\dots,a_{n-1}),a_n)$
		
	\end{itemize}
	
\subsection[Unendlichkeitssatz der Primzahlen]{Satz: Es gibt unendlich viele Primzahlen}

	\textsc{Beweis.} Angenommen es gibt nur endlich viele Primzahlen $p_1,p_2,\dots,p_k$. 
	Dann betrachte die Primfaktorenzerlegung von $n=p_1\cdot p_2\cdot\ldots\cdot p_k +1$. 
	Die Zahlen $p_1,p_2,\dots,p_k$ teilen $n$ \underline{nicht}, sondern lassen den Rest 1. 
	Also sind $p_1,p_2,\dots,p_k$ nicht alle Primzahlen, was im Widerspruch zur Annahme
	steht.\qed

\section{Rechnen mit Brüchen und Reellen Zahlen}

$\mathbb{Q} = \{ \frac{a}{b}| a \in \mathbb{Z}, b \in \mathbb{N_+} \}$ Menge der \textbf{rationalen Zahlen}

\subsection[Rechenregeln für Brüche]{Bemerkung (Rechenregeln für Brüche)}

	Für alle $a,c \in \mathbb{Z}$ und $b,c\in \mathbb{N_+}$ gilt:
	
	\begin{enumerate}
	
	\item (Gleichheit von Brüchen)\\
		\begin{align*}
			\frac{a}{b} = \frac{c}{d} \textnormal{ genau dann wenn } ad=bc
		\end{align*}
		
		Beispiel: $\frac{3}{6}=\frac{1}{2}$ \newline
		
		Kürzen von Brüchen:
		\begin{align*}
			\frac{a\cdot n}{b\cdot n} = \frac{a}{b} \textnormal{ für alle } n \in \mathbb{N_+}
		\end{align*}
		
		
	\item (Addition/Subtraktion von Brüchen)
		\begin{align*}
			\frac{a}{b} + \frac{c}{d} = \frac{ad+bc}{bd} = \frac{a\cdot\tilde{b}+c\cdot \tilde{d}}{kgV(b,d)}
		\end{align*}
		mit $\tilde{b}=\frac{kgV(b,d)}{b}$ und $\tilde{d}=\frac{kgV(b,d)}{d}$. \newline
		
		Beispiele: $\frac{1}{4}+\frac{1}{4}=\frac{2}{4}=\frac{1}{2}, 
		\frac{7}{30}+\frac{11}{45}=\frac{22}{90}+\frac{22}{90}=\frac{43}{90}$
		
	\item (Multiplikation von Brüchen)
		\begin{align*}
			\frac{a}{b} \cdot \frac{c}{d} = \frac{a\cdot c}{b\cdot d}
		\end{align*}			
	
	\item (Division von Brüchen/Doppelbrüche)
		Sei nun $c\neq 0$. Dann gilt:
		
		\begin{align*}
			\frac{\frac{a}{b}}{\frac{c}{d}}= \frac{a}{b}\cdot \frac{d}{c}=\frac{ad}{bc}
		\end{align*}
		
	\item (Kehrwert eines Bruchs)
		\begin{align*}
		\left( \frac{a}{b} \right) ^{-1} = \frac{1}{\frac{a}{b}}=\frac{b}{a}
			\textnormal{ falls } a \in \mathbb{Z}\backslash\{0\}
		\end{align*}
			
	\end{enumerate}
	

\subsection{Beispiele}

	 \begin{enumerate}
	 
	 \item Für $n \geq 1$ gilt
	 \begin{align*}
	 	\frac{1}{m} - \frac{1}{m-1} = \frac{m+1}{m(m+1)}-\frac{m}{m(m+1)} = \frac{1}{m(m+1)},
	 \end{align*}
	  
	  also zB $\frac{1}{3}-\frac{1}{4}=\frac{1}{12}$.
	  
	  \item \begin{itemize}
	  		\item $\frac{1}{2}+\frac{1}{4}=\frac{3}{4}$
	  		\item $\frac{1}{2}+\frac{1}{4}+\frac{1}{8}=\frac{7}{8}$
	  		\item $\frac{1}{2} +\frac{1}{4}+\frac{1}{8}+\frac{1}{16}=\frac{15}{16}$
	  		\item $\frac{1}{2} +\frac{1}{4}+\frac{1}{8}+\dots+\frac{1}{2^n}=\frac{2^n-1}{2^n}$
	  		\end{itemize}
	 
	 \end{enumerate}
	 
	 
\subsection[Potenzen]{Definition (Potenzen)}

	\begin{enumerate}
	\item Sei $a \in \mathbb{R}$. Dann definiere $a^0=1, a^1=a, a^2=a^1\cdot a = a\cdot a$ etc.\\
	Für $n\geq 1$ sei also $a^n= a^{n-1}\cdot a = \underbrace{a \cdot a \cdot \ldots \cdot a}_{n-mal}$.\\
	Die Zahl $a^n$ heißt die $n$-te Potenz von $a$.
	
	\item Sei $a \in \mathbb{R}$ mit $a \neq 0$. Für $n=-k$ mit $k\geq 1$ setze $a^n = a^{-k}= \frac{1}{a^k}$.\\
	
	\end{enumerate}


\subsection{Beispiele}
	
	\begin{itemize}
	\item $343 = 7^3§$
	\item $2^{-3}=\frac{1}{2^3}=\frac{1}{8}=0,125$
	\item $a^{-2}=\frac{1}{a^2}$
	\item $3^6=9^3=729$
	\end{itemize}
	
	
\subsection[Rechenregeln für Potenzen]{Bemerkung (Rechenregeln für Potenzen)}

	Für $a,b \in \mathbb{R}$ und $k,l\in \mathbb{Z}$ gilt:
	
	\begin{enumerate}
	\item $a^k \cdot b^k = (ab)^k$
	\item $a^k \cdot a^l = a^{k+l}$
	\item ${\left(  a^k \right)}^l=a^{kl}$
	\item $\frac{a^k}{a^l}=a^{k-l} \textnormal{ falls } a \neq 0$
	\item ${\left( \frac{a}{b} \right) }^k = \frac{a^k}{b^k} \textnormal{ falls } b \neq 0$
	\end{enumerate}
	
	
\subsection[Wurzeln]{Definition (Wurzeln)}

	\begin{enumerate}
	\item Sei $a \in \mathbb{R_+}=\{a \in \mathbb{R}|a > 0 \}$ und $k\in \mathbb{N_+}$.\\
	Dann gibt es genau ein $b\in \mathbb{R_+}$ mit $b^k=a$. Diese Zahl $b$ heißt die $k$-te Wurzel von $a$ 
	und wird mit $b=\sqrt[k]{a}$ bezeichnet.\\
	Im Fall $k=2$ schreiben wir auch einfach $b=\sqrt{a}$. ("Quadratwurzel") \todo{Gänsefüßchen}
	
	\item Für $a\in \mathbb{R_+}$ und $m,n\in \mathbb{N_+}$ setzen wir $a^{\frac{m}{n}}=\sqrt[n]{a^m}$.
	Insbesondere sei also $a^{\frac{1}{m}}=\sqrt[m]{a}$.\\
	Mit dieser Definition gelten die Rechenregeln für Potenzen auch für rationale Exponenten. Insbesondere
	sei $a^{-\frac{m}{n}}=\frac{1}{a^{\frac{m}{n}}}$.
	\end{enumerate}
	
	
\subsection{Beispiele}

	\begin{itemize}
	\item $\sqrt[3]{24}=\sqrt{2^3\cdot 3}=\sqrt[3]{2^3}\cdot \sqrt[3]{3}=2\cdot \sqrt[3]{3}$
	\item $\sqrt[3]{216}=6$	
	\item $\sqrt{484}=22$
	\item $\sqrt{\frac{36}{121}}=\frac{6}{11}$
	\item $\sqrt{6}\cdot \sqrt{3}= \sqrt{2}\cdot\sqrt{3}\cdot\sqrt{3}=3\sqrt{2}$	
	
	\end{itemize}
	
	
\subsection[Irrationalitätsbeweis von $\sqrt{2}$ nach Euklid]{Satz (Euklid)}
\todo{Blitz, qed}
	\begin{tabular}{ll}
	
	
	\textsc{Behauptung.} 	& $\sqrt{2}$ ist keine rationale Zahl.\\
							&									\\
	\textsc{Beweis.} 		& Angenommen $\sqrt{2}$ wäre rational.\\
							& Dann gäbe es $a,b \in \mathbb{N_+}$ mit $\sqrt{2}=\frac{a}{b}$. \\
							& Durch Kürzen können wir annehmen, dass $ggT(a,b)=1$ gilt. \\
							& Durch Quaddrieren folgt $2= \frac{a^2}{b^2}$, also $2b^2=a^2$. \\
							& Da $a^2$ gerade ist, muss auch $a$ gerade sein, 
							das heißt $\exists c \in \mathbb{N_+}$ mit $a=2c$. \\
							& Einsetzen liefert $2b^2=(2c)^2 \Leftrightarrow b^2=2c^2$. \\
							& Somit muss auch $b$ gerade sein. BLITZ zu $ggT(a,b)=1$.
	\end{tabular}
\section{Rechnen mit Buchstaben}

Seien $a,b,c,\dots$ Buchstabensymbole.\\

\begin{tabular}{ll}
\textsc{Frage.} & Was ist $(x-a) \cdot (x-b) \cdot (x-c) \cdot  \cdots \cdot (x-z)$?\\
\textsc{Hinweis.}& Betrachte den 24. Faktor!
\end{tabular}


\subsection[Definition Term/Koeffizient/Monom/Polynom]{Definition}
	
	\begin{enumerate}
	\item Ein Produkt der Form $(a^{n_a}\cdot b^{n_b}\cdot c^{n_c} \ldots)$ mit $n_a,n_b,n_c,\ldots \in \mathbb{N}$
	 heißt \textbf{Term}.	\\
	 Beachte: $a^2bc=caba=acab$ etc. (Kommutativgesetz)
	 
	 \item Ein Ausdruck der Form $c\cdot t$ mit einem \textbf{Koeffizienten} $c\in \mathbb{R}$ 
	 und einem Term $t$ heißt \textbf{Monom}.
	 \item Eine entliche Summe von Monomen heißt \textbf{Polynom}.	
	
	\end{enumerate}
	
	
\subsection[Rechenregeln für Polynome]{Bemerkung (Rechenregeln für Polynome)}

	Seien $f,g,h,\dots$ Polynome.

	\begin{enumerate}
	\item Distributivgesetze:
		\begin{align*}
		f \cdot(g+h)&=f \cdot g + f \cdot h 
		\textnormal{ (bedeutet } (f \cdot g)+(f \cdot h) \textnormal{ "Punkt vor Strich") } \\
		(f+g)\cdot h &= f \cdot h + g \cdot h
		\end{align*}
	
	\item Kommutativgesetz:
		\begin{align*}
		f \cdot g = g \cdot f,\quad f+g=g+f
		\end{align*}
	\item Assoziativgesetz:
		\begin{align*}
		(f \cdot g) \cdot h = f \cdot (g  \cdot h), \quad (f+g)+h=f+(g+h)		
		\end{align*}
		Die Klammern können auch ganz weggelassen werden.
		
	\item Prioritätsregel: \quad Exponent vor Punkt vor Strich!
		\begin{align*}
		f^2g+h= ((f \cdot f) \cdot g)+h
		\end{align*}
	
	\end{enumerate}
	\todo{Gänsefüßchen}
	
	
\subsection[Beispiele und Formeln]{Beispiele}

	\begin{enumerate}
	\item (Erste binomische Formel)
		\begin{align*}
		(a+b)^2=a^2+2ab+b^2		
		\end{align*}
				
	\item (Zweite binomische Formel)
		\begin{align*}
		(a-b)^2=a^2-2ab+b^2		
		\end{align*}
		
	\item (Dritte binomische Formel)
		\begin{align*}
		(a+b)\cdot (a-b)=a^2-b^2		
		\end{align*}		
		
	\item (Teleskopsumme)			
		\begin{align*}
		1-a^{n+1} = (1 + a + a^2 + a^3 + \cdots + a^n) \cdot (1-a)
		\end{align*}

	\item $1+a^n = (1 - a + a^2 - a^3 + \cdots + a^{n-3} - a^{n-2} + a^{n-1}) \cdot (1+a)$ falls~$n$~ungerade
	\item $a^n - b^n=(a-b)(a^{n-1}+a^{n-2}b+\ldots +ab^{n-2}+b^{n-1})$
	\item $a^n + b^n = (a+b) \cdot (a^{n-1} - a^{n-2} b + a^{n-3} b^2 -+ \cdots b^{n-1})$ falls~$n$~ungerade
	\item $a^3 + b^3 = (a+b) \cdot (a^2 - ab + b^2)$
	
	\end{enumerate}
	
	
\subsection[Rechenregeln für symbolische Berechnungen]{Bemerkung (Rechenregeln für symbolische Berechnungen)}

	\begin{enumerate}
	\item 	\begin{align*}
			(-1)(-1) = 1 \\
			(-1)(+1)=-1 \\
			(-x)(-y)=xy
			\end{align*}
	
	\item (Ausklammern)\\ Man kann die Distributivgesetze oft "andersherum" anwenden:
			\begin{align*}
			ab+a+b+1=a\cdot (b+1)+(b+1)=(a+1)(b+1)\\
			x^2+3x+2= (x+1)(x+2) \quad \textnormal{(\hyperlink{vieta}{Vieta})}		
			\end{align*}
	
	\end{enumerate}\todo{Gänsefüschen, Link to Vieta}

\subsection[Der Grad]{Definition}

	\begin{enumerate}
	\item Ist $t=x_1^{\alpha_1}\cdot \ldots \cdot x_n^{\alpha_n}$ ein Term, so heißt 
	$deg(t)= \alpha_1+ \ldots +\alpha_n$ der \textbf{Grad} von $t$.
	
	\item Ist $f=c_1 t_1 + \ldots + c_s t_s$ ein Polynom mit $c_1 \neq 0, \ldots , c_s \neq 0$ so heißt
	$deg(f)=max\{deg(t_1), \ldots , deg(t_s) \}$ der \textbf{Grad} von $f$.
	
	
	\item Ist $f=c_1 t_1 + \ldots + c_s t_s$ ein Polynom mit $c_1 \neq 0, \ldots , c_s \neq 0$ und gilt
	$deg(t_1)=\ldots=deg(t_s)$, so heißt $f$ ein \textbf{homogenes Polynom}.	
	
	\end{enumerate}


\subsection{Beispiele}

	\begin{itemize}
	\item Das Polynom $f=x^3+y^3$ ist homogen vom Grad 3. 
	\item Das Polynom $p=x^4+4y^4$ ist homogen vom Grad 4.
	
	\end{itemize}
	
	
\subsection[Rationale Funktion]{Definition}

	Seien $f,g$ Polynome mit $g \neq 0$. Dann heißt $\frac{f}{g}$ eine \textbf{rationale Funktion}.
	
	
\subsection{Bemerkung}

	Man kann mit rationalen Funktionen entsprechend der Bruchregeln rechnen.
	
	
\subsection{Beispiele}

	\begin{itemize}
	\item $\frac{1}{x-1}-\frac{1}{x+1}= \frac{(x+1)-(x-1)}{x^2-1}=\frac{2}{x^2-1}$
	\item $\frac{x}{y}-\frac{y}{x}=\frac{x^2-y^2}{xy}$
	\item $\frac{x^2-y^2}{x+y}=\frac{(x-y)(x+y)}{x+y}=x-y$
	\end{itemize}

\section{Lineare und Quadratische Gleichungen}

\subsection[Lineare Gleichungen]{Definition}

	Eine Gleichung der Form $ax+b=0$ mit Zahlen $a,b$ und $a\neq 0$ heißt eine \textbf{lineare Gleichung}
	mit einer Unbestimmten.
		

\subsection{Bemerkung}

	Die Lösung einer Gleichung $ax+b=0$ ist $x_1=-\frac{b}{a}$ (falls $a \neq 0$).\\
	Die Menge $L=\{- \frac{b}{a}\}$ heißt die \textbf{Lösungsmenge} der Gleichung. \\
	(Wenn $\frac{1}{a}$ nicht definiert ist, so gilt $L=\emptyset$.)
	
	
\subsection[Quadratische Gleichungen]{Definition}

	Seien $a,b,c$ Zahlen mit $a\neq 0$. Dann heißt $ax^2+bx+c=0$ eine \textbf{quadratische Gleichung}
	mit einer Unbestimmten.
	

\subsection[Lösen einer quadratischen Gleichung über $\mathbb{R}/\mathbb{C}$]{Bemerkung (Lösen einer quadratischen Gleichung über $\mathbb{R}/\mathbb{C}$)}\todo{R/Q fett}

	\begin{tabular}{@{}ll}
	
	1. Schritt:	& Wegen $a\neq 0$ kann man durch $a$ teilen und erhält: \\
				& $x^2+px+q=0$ mit $p=\frac{b}{a}, q=\frac{c}{a}$ \\
				&\\
	2. Schritt:	& (quadratische Ergänzung) \\
				& $\left( x+\frac{p}{2}\right) ^2 - \frac{p^2}{4}+q=0$ \\
				&\\
	3. Schritt:	& (Wurzel ziehen) \\
				& $\left( x+\frac{p}{2}\right) ^2 = \frac{p^2}{4}-q=\frac{p^2-4q}{4}$ \\
				& Ist $p^2-4q <0$, so gibt es in $\mathbb{R}$ keine Lösung. \\
				& Ansonsten: $x+\frac{p}{2}=\pm \frac{1}{2} \sqrt{p^2-4q}$ \\
				&\\
				& Die Lösungen sind also \\
				& $x_1=-\frac{p}{2}+\frac{1}{2}\sqrt{p^2-4q}$ und $x_1=-\frac{p}{2}-\frac{1}{2}\sqrt{p^2-4q}$ 	
	
	\end{tabular}\\
	 ~\newline\newline
	Die Zahl $\Delta=p^2-4q$ heißt die \textbf{Diskriminante} der Gleichung.
	
	

\subsection[Satz von Vieta]{Satz (Vieta)}\hypertarget{vieta}{}

	Seien $x_1.x_2$ die Lösungen einer quadratischen Gleichung $x^2+px+q=0$.\\
	Dann gilt: $x_1+x_2=-p$ und $x_1 \cdot x_2=q$.\\ \newline
	\textsc{Beweis.} Sind $x_1, x_2$ die Lösungen, so gilt:
	\begin{align*}
	&(x-x_1)(x-x_1)=0 \textnormal{ und somit } x^2-x_1 x-x_2 x+x_1 x_2=0,\\
	&\textnormal{also } x^2-(x_1 + x_2)x+(x_1 x_2)=0.
	\end{align*}\todo{centering, qed}
	\newline 
	\textsc{Anwendung:} Um $x^2+px+q=0$ zu lösen, finde zwei Zahlen mit Summe $-p$ und Produkt q.
	
	
\subsection{Beispiele}

	\begin{itemize}
	\item $x^2-3x+2=0$ hat die Lösungen $x_1=1$ und $x_2=2$.
	\item $x^2-4x+3, L=\{1,3\}$
	\item $x^2+3x+2, L=\{-1,-2\}$
	\item $x^2+x-2, L=\{1,-2\}$
	\end{itemize}
	
	
\subsection[Substitution]{Bemerkung (Substitution)}

	Manchmal kann man eine Gleichung durch eine geschickte \textbf{Substitution} lösen.
	
	\begin{itemize}
	\item 	Löse $x^4-7x^2+10$ in $\mathbb{R}$. Setze $y=x^2$.\\
			Erhalte $y^2-7y+12=0$ mit $L=\{3,4\}$ und somit $x_{1/2}=\pm \sqrt{3}, x_{3/4}=\pm 2$.
			
	\item	Löse $x-18\sqrt{x}+17=0$ in $\mathbb{R}$. Setze $y=\sqrt{x}$.\\
			Erhalte $y^2-18y+17=0$ mit $L=\{1,17\}$, also $\sqrt{x}=1$ und $\sqrt{x}=17$.
			Somit sind $x_1=1$ und $x_2=289$.
		
	\end{itemize}
	
	
\subsection[Lineare Gleichungssysteme]{Bemerkung (Lineares Gleichungssystem)}

	Gegeben seien Zahlen $a_1,a_2,b_1,b_2,c_1,c_2$ mit $a_1 b_1-a_2 b_2 \neq 0$.\\
	Dann heißt $\left\{  \begin{array}{l}
                  			a_1x+b_1 y+c_1=0\\
                  			a_2x+b_2 y+c_2=0
                			\end{array}
                \right.$ ein \textbf{lineares Gleichungssystem} mit zwei Unbestimmten $x,y$.
                
    \begin{enumerate}
    \item Lösungsmethode "Einsetzen"\\
    Ist $a_1 \neq0$, so wird $x=-\frac{b_1}{a_1}y-\frac{c_1}{a_1}$. Setze dies in die zweite Gleichung ein und
    erhalte $a_2\left( -\frac{b_1}{a_1}y-\frac{c_1}{a_1}  \right)+b_2 y+c=0$. Löse diese lineare Gleichung 
    und erhalte $y_1$. Dann gilt $x_1=-\frac{b_1}{a_2}y_1-\frac{c_1}{a_1}$. $L=\{(x_1,y_1)\}$.\\ \newline
    
   	\underline{Sonderfall:} $y$ hebt sich in der ersetzten Gleichung auf:\\
    $a_2\cdot \left( - \frac{b_1}{a_1} \right)+b_2=0$, also $ \frac{-a_2 b_1+a_1 b_2}{a_1}=0$ und somit
    $a_1 b_2 - a_2 c_1=0$.\\ \newline
    In diesem Fall lautet die ersetzte Gleichung:\\
    $a_2 \left( -\frac{c_1}{a_1}\right)+c_2=0$, also $\frac{-a_2 c_1+a_1 c_1}{a_1}=0$ und somit
    $a_1 c_2-a_2 c_1=0$\\ \newline
    Es gibt zwei Möglichkeiten:
    	\begin{enumerate}
		\item $a_1 c_2 - a_2 c_1 \neq 0 \Rightarrow L=\emptyset$
		\item $a_1 c_2 - a_2 c_1 = 0 \Rightarrow y$ beliebig, $x=-\frac{b_1}{a_1}y-\frac{c_1}{a_1}$\\
		Somit gilt: 
		$L=\left\{\left( -\frac{b_1}{a_1}\cdot \lambda-\frac{c_1}{a_1},\lambda \right)\big\vert 
		\lambda\in \mathbb{R} \right\} \subseteq \mathbb{R}^2$  
    	
    	\end{enumerate}
  
  	\item Lösungsmethode "Inderreduzieren", "Gauß-Verfahren"\\
  	\underline{Ziel:} Bilde Linearkombinationen der beiden Gleichungen, 
  	in denen nur eine der beiden Unbestimmten vorkommt.
  
    \end{enumerate}\todo{Gänsefüßchen}
              

\subsection{Beispiel}

	Löse  $\left\{  \begin{array}{ll}
                  			2x+5y=9 & \textnormal{(I)}\\
                  			3x-4y=2 & \textnormal{(II)}
                			\end{array}
                \right.$\\
                
    $3\cdot$(I)$-2\cdot$(II):  $15y+8y=27-4$ liefert $y=1$.\\
    Einsetzen von $y=1$ in (II) ergibt $3x=6 \Leftrightarrow x=2 \Rightarrow L=\{(2,1)\}$.
    
    
\subsection[Schnittpunkt von zwei Kreisen]{Beispiel (Schnittpunkt von zwei Kreisen)}
	
	
	\begin{tabular}{@{}lrl}
	
	Zwei Kreise seien gegeben durch: 	& $\left.  \begin{array}{l}
                  								 		x^2+y^2+2x-6y+1=0 \\
                  										(x+1)^2+(y-3)^2=9
                									\end{array}
             						 	  \right\}$ & $K_1$\\.

										&&\\
										&  $x^2+y^2-4x-4y=0 \quad$ &$K_2$
 
	\end{tabular}	\newline
	
	Gleichung $K_1 - K_2$:  $6x-2y+6=0$, also $y=3x+3$.\\
	Setze dies in $K_1$ (oder $K_2$) ein: $x^2+(3x+3)^2+2x-6\cdot (3x+1)+1=0$.\\
	Liefert: $x_1=-1,\ x_2=\frac{4}{5}$, also $y_1=0,\ y_2=\frac{27}{5} \Rightarrow
	L=\left\{(-1,0), \left(\frac{4}{5},\frac{27}{5} \right) \right\}$


\subsection[Aufgabe]{Aufgabe (Aus einem alten chinesischem Rechenbuch)}

	In einem Stall sind Hühner und Schweine.\\
	 Es sind 40 Tiere. Zusammen haben sie 70 Füße.\\ \newline 
	Wie viele Tiere von jeder Sorte sind es?
	
\section{Ungleichungen}

Seien $f,g$ Polynome in Unbestimmten $x_1,x_2,\dots,x_n$ (oder $x,y,z$) mit Koeffizienten aus $\mathbb{R}$.

\subsection{Definition}

	Es gibt 5 Typen von Ungleichungen:
	
	\begin{enumerate}
	\item $f\leq g$
	\item $f\geq g$
	\item $f<g$
	\item $f>g$
	\item $f\neq g$
	\end{enumerate}
	
	\underline{Interpretation:} $f\leq g$ bedeutet, dass die Ungleichung gelten soll, wenn man für $x_1,\dots,x_n$
	Zahlen (aus einem Definitionsbereich $D\subseteq \mathbb{R}^n$) einsetzt.
	
	
\subsection{Beispiele}

	\begin{itemize}
	\item Für $x\in \mathbb{R}$ gilt: $x^2\geq 0$.
	\item Für $x,y \in \mathbb{R}$ gilt: $x^2+y^2\geq 2xy$\\ \newline
	\textsc{Beweis.}\begin{align*}
					(x-y)^2 &\geq 0 \\
					\Leftrightarrow x^2 -2xy +y^2 &\geq 0 \\
					\Leftrightarrow x^2+y^2 &\geq 2xy
					\end{align*}\\
	\textsc{Folgerung:} Für $x,y\geq 0$ gilt: $\quad  \underbrace{\sqrt{xy}}_{\textnormal{geometrisches Mittel}}
						\leq \underbrace{\sqrt{\frac{x^2+y^2}{2}}}_{\textnormal{quadratisches Mittel}}$
						
	\item Arithmetisches Mittel:  $\frac{x+y}{2}\leq \sqrt{xy}$		
	
	\end{itemize}\todo{Größenverhätnisse}


\subsection[Rechenregeln für Ungleichungen]{Bemerkung (Rechenregeln für Ungleichungen)}

	\begin{enumerate}
	\item 	\begin{itemize}
			\item $f \leq g$ ist äquivalent mit $g \geq f$
			\item $f<g$ ist äquivalent mit $g>f$
			\item $f \leq g$ ist äquivalent mit $[f<g$ oder $f=g]$
			\item $f \neq g$ ist äquivalent mit $[f<g$ oder $f>g]$
			\end{itemize}	
	
	\item Sei $h$ ein weiteres Polynom.\\ Dann ist $f \leq g$ äquivalent mit $f+h \leq g+h$.	
	\item $f \leq g$ ist äquivalent mit $-f\geq -g$
	\item Gilt $f \leq g$ und $h \geq 0$ so folgt $f \cdot h \leq g \cdot h$ \\
	Gilt $f \leq g$ und $h \leq 0$ so folgt $f \cdot h \geq g \cdot h$
	\item Für $0<f \leq g$ gilt $0 < \frac{1}{g} \leq \frac{1}{f}$
	
	\end{enumerate} 

	Eine Ungleichung zu lösen bedeutet, alle $(x_1,\dots,x_n)\in \mathbb{R}^n$ zu finden, 
	für die die Ungleichung gilt.
	
	
\subsection{Beispiel}

	Löse  $\left\{  \begin{array}{ll}
                  			3x-4y \leq 1	& \textnormal{(I)}\\
                  			x+y \geq 2		& \textnormal{(II)}
                			\end{array}
                \right.$.\\ \newline \newline
     \underline{Skizze:}\todo{Skizze}
     
     
	$\left.  \begin{array}{ll}
             
                \textnormal{(II)':}			& -x-y\geq -2								\\
                \textnormal{(I)+3(II)':}	& -7y \leq 5 \Rightarrow y \geq \frac{5}{7} \\ \hline
                \textnormal{aus (II):}		& x \geq 2-y								\\ \hline
     			\textnormal{aus (I):}		& y\leq \frac{1}{3}+\frac{4}{3}y                
                
              \end{array} 	  \right\}$ Es folgt: $L=\left\{ \left( x,y \right) \in \mathbb{R}^2 
              | y\geq \frac{5}{7} \land 2-y \leq x \leq \frac{1}{3}+\frac{4}{3}y \right\}$
              
              
\subsection{Bemerkung}

	Ist $h: \mathbb{R} \rightarrow \mathbb{R}$ monoton steigend, (das heißt aus $x\leq y$ folgt $h(x)\leq h(y)$,)
	so gilt:
	
	\begin{align*}
	\textnormal{Aus } f \leq g \textnormal{ folgt } h &\circexpl{Komposition} f \leq h \circ g.
	\end{align*}	
	
	
\subsection{Beispiele}

	\begin{itemize}
	\item Die Funktion $h: \mathbb{R}_0^+ \rightarrow \mathbb{R}_0^+, x\mapsto\sqrt{x}$ ist monoton steigend.\\
	Somit folgt aus $0 \leq f \leq g$ die Ungleichung $0 \leq \sqrt{f} \leq \sqrt{g}$.
	
	\item Die Abbildung $\ln: \mathbb{R}_+ \rightarrow \mathbb{R}, x\mapsto\ln(x)$ ist monoton steigend.\\
	Aus $0 \leq f \leq g$ die Ungleichung $0 \leq \ln(f) \leq \ln(g)$.
	\end{itemize}
	
	
\subsection{Beispiel}

	Löse die Ungleichung $x^2-\frac{1}{2}x-\frac{1}{2} \geq 0$ in $\mathbb{R}$.\\ \newline
	Quadratische Ergänzung:
	\begin{align*}
	\left( x - \frac{1}{4} \right) ^2 \geq \frac{1}{2} + \frac{1}{16} = \frac{9}{16}
	\end{align*}
	\underline{Fallunterscheidung!}\\
	
	\begin{enumerate}
	\item Fall: $x-\frac{1}{4} \geq 0$: Wurzelziehen ist erlaubt.\\
	...und liefert: $x-\frac{1}{4}$, also $x\geq 1$.\\ \newline
	Die Lösungsmenge im 1. Fall ist also:\\
	$L_1=\left\{x\in\mathbb{R}|x\geq\frac{1}{4} \land x\geq 1\right\}=\{x\in\mathbb{R}|x\geq 1\}$
	
	\item Fall: $x-\frac{1}{4}<0$, also $\frac{1}{4}-x>0$\\
	Die Ungleichung $(\frac{1}{4}-x)^2\geq \frac{9}{16}$ liefert 
	$\frac{1}{4}-x\geq\frac{3}{4}$, also $x\leq -\frac{1}{2}$\\ \newline
	\textit{Anmerkung des Autors: in der Klammer wurde -1 ausgeklammert, da diese beim Quadrieren belanglos ist.
	Eine (einfachere) Alternative ist \ref{betrag}.}\\
	\newline Dies zeigt $L_2=\left\{ x\in\mathbb{R}|x<\frac{1}{4} \land x\leq -\frac{1}{2}\right\} =
	\left\{ x\in \mathbb{R}|x\leq -\frac{1}{2}\right\}$
	\end{enumerate}
	~\newline
	Insgesamt ergibt sich die Lösungsmenge
	$L=L_1\cup L_2=\left\{ x\in\mathbb{R}|x\geq 1\lor x\leq -\frac{1}{2} \right\}$.
	
	
\subsection[Betrag]{Definition}

	Für jedes $x\in \mathbb{R}$ heißt\\ 
	
	$\vert x \vert \left\{  \begin{array}{cl}
                  			x 	&\textnormal{ für } x \geq 0\\
                  			-x 	&\textnormal{ für } x<0
                			\end{array}
                \right.$     der \textbf{(Absolut-)Betrag} von $x$.
                
                
\subsection{Beispiel}\label{betrag}

	Im letzten Beispiel folgt aus $\left( x-\frac{1}{4} \right)^2$ die Ungleichung 
	$\vert x-\frac{1}{4}\vert \geq \frac{3}{4}$
	
	
\subsection[Betragsungleichungen]{Beispiel}

	Löse die Ungleichung $\vert x+1 \vert + \vert x-1 \vert \leq 2$.\\ 
	
	\underline{Fallunterscheidung!} \newline
	
	\begin{enumerate}
	\item Fall: $x<-1$\\
	Die Ungleichung lautet $-(x+1)-(x-1)\leq 2$ und somit $x \geq -1$.\\
	Dies liefert $L_1=\emptyset$
	\item Fall: $-1 \leq x<1$\\
	Die Ungleichung lautet $(x+1)-(x-1)\leq 2$ und somit $2 \leq 2$.\\
	Somit folgt $L_2=\{ x\in \mathbb{R}|-1\leq x\leq 1 \}$
	\item Fall: $x\geq 1$\\
	Die Ungleichung lautet $(x+1)+(x-1)\leq 2$ und somit $x \leq 1$.\\
	Dies zeigt $L_3=\{1\}$.
	
	\end{enumerate}
	~\newline
	Insgesamt erhalten wir $L=L_1\cup L_2\cup L_3 = \{x\in\mathbb{R}|-1\leq x \leq 1 \}$.
	
	
\subsection{Dreiecksungleichung}

	\begin{enumerate}
	\item Für $x,y \in \mathbb{R}$ gilt: $\vert x+y \vert \leq \vert x \vert + \vert y \vert$
	(\textbf{Dreiecksungleichung})
	\item Es gilt für alle $x,y \in \mathbb{R}$ die \textbf{umgekehrte Dreiecksungleichung}:
	$\big\vert \vert x \vert - \vert y \vert \big\vert \leq \vert x+y\vert$
	\end{enumerate}
	
	\textsc{Beweis.}
	\begin{enumerate}
	\item Aus $xy \leq \vert x\vert \cdot \vert y\vert = \vert xy \vert$ folgt
	$x^2+2xy+y^2 \leq {\vert x \vert}^2+2\vert x \vert \vert y \vert + {\vert y \vert}^2$,
	also $(x+y)^2 \leq (\vert x \vert + \vert y \vert)^2$.\\
	Da $\vert x+y\vert, \vert x \vert$ und $\vert y \vert$ nicht negativ sind, ist Wurzelziehen erlaubt.
	
	\item Nach 1. gilt $\vert x \vert \leq \vert x+y\vert + \vert -y \vert = \vert (x+y)-y \vert$\\
	und somit $\vert x+y \vert \geq \vert x \vert - \vert y \vert$.\\
	Andererseits gilt, ebenfalls nach 1., die Ungleichung\\ 
	$\vert x+y-x\vert = \vert y \vert \leq \vert x+y \vert + \vert -x \vert = \vert x+y \vert + \vert x \vert$\\
	und somit $\vert x+y \vert \geq \vert y \vert - \vert x \vert$.\\ \newline
	Kombiniert man beide Erkenntnisse, so folgt 
	$\vert x+y \vert \geq \big\vert \vert x \vert - \vert y \vert\big \vert$.
	\end{enumerate}\todo{qed}
	
	
\subsection{Beispiel}

	Löse $\sqrt{2x-1} < x+1$ für $x\in \mathbb{R}$.\\ \newline
	Damit die Wurzel definiert ist, muss gelten $2x-1\geq 0$, also $x\geq \frac{1}{2}$.\\
	Dann ist die rechte Seite positiv und Quadrieren erlaubt.\\ \newline
	Es folgt: $2x-1<x^2+2x+1$, also $x^2>-2$.\\ \newline
	Insgesamt erhalten wir: $L=\{ x\in \mathbb{R}| x\geq 0,5\}$.
	
	
\subsection{Beispiel}

	Löse $\sqrt{x^2+1}>x+1$ in $\mathbb{R}$.\\
	
	\begin{enumerate}
	\item Fall: $x+1<0$.\\
	In diesem Fall gilt $\sqrt{x^2+1}>0>x+1$,\\
	also $L_1=\{x\in\mathbb{R}|x<-1\}$.
	\item Fall: $x+1\geq 0$.\\
	Jetzt ist Quadrieren eine Äquivalenzumformung und es folgt $x^2+1>x^2+2x+1$, also $x<0$.\\
	Dies liefert $L_2=\{x\in\mathbb{R}|-1\geq x <0 \}.$
	\end{enumerate}
	~\newline
	Insegesamt folgt: $L=L_1\cup L_2 = \{x\in \mathbb{R}|x<0\}$.

\section{Ebene Geometrie}
	
	Die Grundlage der ebenen Geometrie ist die \textbf{Zeichenebene}. Nach der Einführung von rechtwinkligen
	(kartesischen) Koordinaten ist dies die Menge~$\mathbb{R}^2$. Genauer gesagt ist sie der 
	\textbf{2-dimensionale affine Raum} über $\mathbb{R}$ und wird mit $\mathbb{A}^2(\mathbb{R)}$ bezeichnet.\\
	Die Elemente von $\mathbb{A}^2(\mathbb{R)}$ heißen \textbf{Punkte}.\\
	Zu je zwei verschiedenen Punkten in	$\mathbb{A}^2(\mathbb{R)}$ gibt es genau eine \textbf{Gerade},
	die diese enthält.\\
	\newline
	\underline{Notation:}
	\begin{itemize}
	\item $A,B,\dots\in\mathbb{A}^2(\mathbb{R)}$ Punkte
	\item $G=AB$ Gerade durch die Punkte $A$ und $B$
	\item $[AB]$ Strecke von $A$ nach $B$
	\item $a=\overline{AB}$ Länge der Strecke $[AB]$
	\end{itemize}
	
	
\subsection{Definition}

	Seien $A,B,C\in\mathbb{A}^2(\mathbb{R)}$ drei \textbf{nicht kollineare} (=nicht auf einer Gerade) Punkte.\\
	Dann heißt die Vereinigung der Strecken $[AB],[BC]$ und $[CA]$ das \textbf{Dreieck} mit den
	\textbf{Seiten(längen)} $a=\overline{BC},b=\overline{AC},c=\overline{AB}$. \todo{Skizze}
	Die Punkte $A,B,C$ heißen die \textbf{Ecken} des Dreiecks.\\ \newline
	\underline{Notation:} $\Delta ABC$ (Nummerierung \underline{gegen} den Uhrzeigersinn!)\\
	\newline
	Die Winkel an den Ecken des Dreiecks werden mit $\alpha =\angle BAC, \beta =\angle CBA, \gamma =\angle ACB$
	bezeichnet.
	
	
\subsection[Das gleichseitige Dreieck]{Beispiel (Das gleichseitige Dreieck)}

	Es gelte: $a=b=c$. Dann heißt $\Delta ABC$ ein \textbf{gleichseitiges Dreieck}.\todo{Skizze}
	
	
\subsection[Der Innenwinkelsummensatz]{Satz}

	In einem Dreieck $\Delta ABC$ gilt $\alpha+\beta+\gamma=180^{\circ}=\pi$.
	
	
\subsection[Das Bogenmaß]{Definition (Bogenmaß)}

	Im \textbf{Bogenmaß} gilt $180^{\circ}=\pi,360^{\circ}=2\pi,90^{\circ}=\frac{\pi}{2}$ usw.\\
	Es entspricht der Länge des Kreisbogens auf dem Einheitskreis, der diesem Winkel entspricht.
	
	
\subsection{Beispiel}

	Im gleichseitigen Dreieck gilt: $\alpha=\beta=\gamma=60^{\circ}=\frac{\pi}{3}$.
	
	
\subsection[Das gleichschenklig-rechtwinklige Dreieck]{Beispiel}

	Das\textbf{gleichschenklig-rechtwinklige Dreieck} erfüllt $\alpha=\beta=45^{\circ}$ und $\gamma=90^{\circ}$.\\
	Ein $90^{\circ}$-Winkel heißt auch \textbf{rechter Winkel} und wird auch wie folgt notiert: \todo{Skizze}
	
	
\subsection{Bemerkung}

	\begin{enumerate}
	\item Die Winkel an einer Geradengleichung erfüllen \textsc{Skizze} mit $\alpha+\beta=180^{\circ}$.
	\item Werden zwei \textbf{parallele} Geraden (d.h. Geraden, die sich nicht schneiden) von einer dritten
	Gerade geschnitten, so gilt: \textsc{Skizze} ("Z-Winkelsatz") $\alpha+\beta=180^{\circ}$
	\end{enumerate}\todo{Skizze, Gänsefüßchen}
	
	
\subsection[Der allgemeine Innenwinkelsummensatz]{Satz}

	Die Summe der Innenwinkel in einem \textbf{konvexen} $n$-Eck (d.h. die Strecke von einem Punkt des $n$-Ecks
	zu einem anderen liegt vollständig im Inneren) beträgt $(n-2)\cdot 180^{\circ}$.\\
	\newline
	\textsc{Beweis.}\todo{Skizze}\\
	Ausgehend von einer Ecke $A$ zeichne die Diagonalen zu den nicht anliegenden Ecken ein. Diese zerlegen das
	$n$-Eck in $n-2$ Dreiecke. Aus der Aufteilung der Außenwinkel folgt, dass die Winkelsumme im $n$-Eck gleich
	der $n-2$ Dreiecke ist und es folgt die Behauptung.\todo{qed}
	
	
\subsection[Das reguläre $n$-Eck]{Beispiel (Das reguläre $n$-Eck)}

	Das reguläre $n$-Eck hat $n$ Ecken, $n$ gleich lange Seiten und $n$ gleich große Innenwinkel.\todo{Skizze}
	
	
\subsection[Besondere Linien im Dreieck]{Definition (Besondere Linien im Dreieck)}

	\begin{enumerate}
	\item Fällt man von einer Ecke das Lot auf die gegenüberliegende Gerade, so heißt die entstehende Stecke
	eine \textbf{Höhe} des Dreiecks.\\
	\underline{Notation:} $h_a,h_b,h_c$
	
	\item Verbindet man eine Ecke des Dreiecks mit dem Mittelpunkt der gegenüberliegenden Seite, so heißt die
	entstehende Strecke eine \textbf{Seitenhalbierende}.\\
	\underline{Notation:} $s_a,s_b,s_c$
	
	\item Fällt man auf die Mittelpunkte der Seiten einers Dreiecks die Lote, so heißen diese die
	\textbf{Mittelsenkrechten} des Dreiecks.\\
	\underline{Notation:} $M_a,M_b,M_c$
	
	\item Zeichnet man im Dreieck $\Delta ABC$ jeweils die Geraden ein, die die Winkel $\alpha,\beta,\gamma$
	halbieren, bezeichet man diese als \textbf{Winkelhalbierende}.\\
	\underline{Notation:} $W_{\alpha},W_{\beta},W_{\gamma}$
	\end{enumerate}\todo{Skizzen}


\subsection[Kongruente Dreiecke]{Definition (Kongruente Dreiecke)}	

	\begin{enumerate}
	\item Zei Teilmengen $T_1,T_2$ von $\mathbb{A}^2(\mathbb{R})$ heißen \textbf{kongruent} 
	(oder \textbf{deckungsgleich}), wenn sie durch eine \textbf{Bewegung} (hier: durch Verschiebungen,
	Rotationen, Spiegelungen) ineinander überführbar sind. Statt Bewegung sagt man auch oft
	\textbf{Kongruenzabbildung}.
	
	\item Eine Abbildung $\varphi:\mathbb{A}^2(\mathbb{R})\rightarrow\mathbb{A}^2(\mathbb{R})$ heißt 
	\textbf{Kongruentabbildung}, wenn wie Längen und Winkel erhält.
	
	\end{enumerate}
	
	
\subsection[Die Kongruenzsätze im Dreieck]{Satz (Die Kongruenzsätze im Dreieck)}	
	
	Zwei Dreiecke $\Delta ABC$ und $\Delta A'B'C'$ sind kongruent, wenn eine der folgenden Bedingungen erfüllt
	ist:
	
	\begin{enumerate}
	\item (sss-Satz) Die Seitenlängen sind paarweise gleich.
	\item (sws-Satz) Zwei Seitenlängen und der eingeschlossene Winkel sind jeweils gleich.
	\item (wsw-Satz) Zwei entsprechende Seitenlängen und die jeweils anliegenden Winkel sind gleich.
	\end{enumerate}
	
	
\subsection[Besondere Punkte im Dreieck]{Satz}

	\begin{enumerate}
	\item Die drei Mittelsenkrechten eines Dreiecks schneiden sich in einem Punkt. Der Schnittpunkt ist der 
	\textbf{Umkreismittelpunkt} des Dreiecks.
	\item Die Winkelhalbierenden schneiden sich in einem Punkt. Der Schnittpunkt ist der
	\textbf{Innkreismittelpunkt} des Dreiecks.
	\item Die drei Seithenhalbierenden eines Dreiecks schneiden sich in einem Punkt. dieser Punkt ist der
	\textbf{Schwerpunkt} des Dreiecks.
	\item Die drei Höhen des Dreiecks schneiden sich in einem Punkt ("\textbf{Höhenschnittpunkt}").	
	\end{enumerate}\todo(Skizze,Gänsefüßchen,Beweis)
	
	
\subsection[Die Fläche des Dreiecks]{Satz (Fläche des Dreiecks)}

	Die Fläche $F$ des Dreiecks $\Delta ABC$ ist gegeben durch $F=\frac{1}{2}\cdot c \cdot h_c$.\todo{Skizze}\\
	\newline
	\textsc{Beweis.} $F=F_1+F_2=\frac{1}{2}\cdot p \cdot h_c + \frac{1}{2}\cdot q \cdot h_c =
	\frac{1}{2}h_c(p+c)=\frac{1}{2}\cdot c\cdot h_c$\todo{qed}
	
	
\subsection[Das rechtwinklige Dreieck]{Beispiel (Rechtwinkliges Dreieck)}

	Ist in einem Dreieck einer der Innenwinkel gleich $90^{\circ}$, so heißt es ein 
	\textbf{rechtwinkliges Dreieck}. Dann heißt $[AB]$ die \textbf{Hypotenuse} des Dreiecks und $[BC],[AC]$ heißen 
	die \textbf{Katheten} des Dreiecks.\\
	Es gilt: $\alpha+\beta=90^{\circ}$.
	
	
\subsection[Der Satz des Thales]{Satz (Der Satz des Thales)}

	Ist $M$ der Mittelpunkt der Hypotenuse und zeichnet man den Kreis um $M$ mit Radius $\frac{c}{2}$, so liegt
	die Ecke $C$ auf diesem Kreis. Er heißt \textbf{Thaleskreis} des Dreiecks.\\
	\newline
	Umgekehrt gilt: Liegt die Eckt $C$ auf dem Kreis um den Mittelpunkt $M$ von $[AB]$ mit Radius $\frac{c}{2}$,
	so ist das Dreieck $\Delta ABC$ rechtwinklig mit $\gamma=90^{\circ}$.
	\todo{Beweise+Skizzen}
	
	
\subsection[Die zentrische Streckung]{Definition (Zentrische Streckung)}	
	
	Eine \textbf{zentrische Streckung (Homothetie)} mit Zentrum $Z$ und Streckungsfaktor $\lambda\in\mathbb{R}_+$
	ist die Abbildung, die jedem Punkt $P$ abbildet auf den Punkt $Q$ auf der Halbgeraden $[ZP$, der von $Z$ (in
	Richtung $P$) die Entfernung $\lambda\cdot \overline{ZP}$ besitzt. \todo{Skizze}\\
	\newline
	Zentrische Streckungen sind winkelerhaltend.
	
	
\subsection[Der Strahlensatz]{Satz (Der Strahlensatz)}

	Gegeben seien zwei Geraden, die sich in einem Punkt $Z$ schneiden. Diese Geraden $G_1,G_2$ werden von zwei
	parallelen Geraden $H_1,H_2$ geschnitten. Die Schnittpunkte seien $A,A',B,B'$.\todo{Skizze}\\
	\newline
	
	\begin{tabular}{cccc}
	Setze	& $a=\overline{ZA}$	& $a'=\overline{ZA'}$	& $a''=\overline{AA'}$\\
			& $b=\overline{ZB}$	& $b'=\overline{ZB'}$	& $b''=\overline{BB'}$\\
			& $c=\overline{AB}$	& $c'=\overline{A'B'}$	&
	\end{tabular}
	~\newline
	\begin{enumerate}
	\item Strahlensatz
	\begin{align*}
	\frac{a}{a'}=\frac{b}{b'} \textnormal{ und } \frac{a}{a''}=\frac{b}{b''}
	\end{align*}
	
	\item Strahlensatz
	\begin{align*}
	\frac{a}{a'}=\frac{b}{b'}=\frac{c}{c'}
	\end{align*}
	
	\item Umkehrung des 1. Strahlensatz
	\begin{align*}
	\textnormal{Gilt } \frac{a}{a'}=\frac{b}{b'} \textnormal{ so folgt } H_1\parallel H_2.
	\end{align*}
	\end{enumerate}
	
	Vorsicht: Die Umkehrung des 2. Strahlensatzes gilt \underline{nicht}.\todo{Blitz}\\
	\newline\newline
	
	
	\textsc{Beweis.}
	\begin{enumerate}
	\item Betrachte die zentrische Streckung mit dem Zentrum $Z$ und dem Streckungsfaktor $\frac{a'}{a}$.\\
	Bei der zentrischen Streckung wird $H_1$ in eine parallele Gerade abgebildet, denn würde sich $H_1$ und ihr
	Bild in einem Punkt schneiden, so wäre dieser Punkt ein Fixpunkt ungleich $Z$ der Abbildung, was für
	$\lambda\neq 1$ nicht geht.\\
	Also wird $H_1$ auf $H_2$ abgebildet. Somit wird $B$ auf $B'$ abgebiltet und 
	$\frac{b'}{b}=\lambda=\frac{a'}{a}$.	
	
	\item Die Strecke $[AB]$ wird bei der zentrische Streckung auf $[A'B']$ abgebildet, also gilt
	$\frac{c'}{c}=\lambda=\frac{a'}{a}$.
	
	\item Nach Voraussetzung wird bei der zentrischen Streckung mit $\lambda=\frac{a'}{a}$ sowohl $A$ in $A'$
	also auch $B$ in $B'$ überführt. Also wird $[AB]$ in $[A'B']$ überführt und es folgt $A'B'\parallel AB$.
	\end{enumerate}\todo{qed}
	
	
\subsection[Ähnlichkeit]{Definition}

	Zwei ebene geometrische Figuren heißen \textbf{ähnlich}, wenn es eine Kongruenzabbildung und eine
	anschließende zentrische Streckung gibt, die die eine in die andere überführt.
	
	
\subsection{Korollar}

	Sind zwei ebene geometrische Figuren ähnlich, so sind entsprechende Winkel gleich und entsprechende
	Streckenlängen stehen in einem festen Verhältnis $\lambda$.
	
	
\subsection{Beispiel}

	Der Schwerpunkt eines Dreiecks teilt die Seitenhalbierenden im \mbox{Verhältnis $2:1$}.\\
	\newline
	\textsc{Beweis.}\todo{Skizze}\\
	Zeichne die Parallele zu $CM_c$ durch $M_a$. Ihr Schnittpunkt mit der Seite $[AB]$ sei $P$.\\
	Nach dem Strahlensatz gilt: $\frac{\overline{BM_c}}{\overline{BP}}=\frac{\overline{BC}}{BM_a}=\frac{2}{1}
	=\frac{\overline{BS}}{BQ}$.\\
	Strahlensatz mit Zentrum $A$: Die Geraden $AM_a$ und $AB$ werden von zwei paralleln Geraden geschnitten:
	$\frac{\overline{AM_a}}{\overline{AS}}=\frac{\overline{BC}}{\overline{BM_a}}=\frac{3}{2}$\todo{WTF?!? qed}
	
	
\subsection[Die Satzgruppe des Pythagoras]{Satz (Die Satzgruppe des Pythagoras)}

	Sei $\Delta ABC$ ein rechtwinkliges Dreieck mit $\gamma=90^{\circ}$.
	\begin{enumerate}
	\item (Satz des Pytagoras, ca. 540 v. Chr.)\\
	\begin{align*}
	a^2+b^2=c^2
	\end{align*}
	
	\item (Umkehrung des Satzes von Pytagoras)\\
	Gilt in einem Dreieck $\Delta ABC$ die Gleichung $a^2+b^2=c^2$, so ist $\Delta ABC$ rechtwinklig mit 
	$\gamma = 90^{\circ}$.
	
	\item (Höhensatz)\\
	\begin{align*}
	h^2=pq, \textnormal{ wobei } h=h_c \textnormal{ und } p,q \textnormal{ \textbf{Hypotenusenabschnitte}}.
	\end{align*}
	
	\item (Kathetensatz)
	\begin{align*}
	a^2=cq \textnormal{ und } b^2=cp
	\end{align*}
	\end{enumerate}\todo{Skizze}
	~\newline\newline
	
	\textsc{Beweis.}
	
	\begin{enumerate}
	\item Die äußeren Dreiecke sind kongruent.
	\begin{align*}
	\Rightarrow (a+b)^2=c^2+4(\frac{1}{2}ab)=c^2+2ab\Rightarrow a^2+b^2=c^2
	\end{align*}
	
	\item Konstruiere das rechtwinklige Dreieck $\Delta A'B'C'$ mit den Seitenlängen $a.b.c$ und Katheten $a,b$.\\
	Nach 1. gilt: $\tilde{c}=c$. Nach sss-Satz folgt die Behauptung.
	
	\item Nach 1. gilt: $p^2+h^2=b^2$ und $q^2+h^2=q^2$
	\begin{align*}
	\Rightarrow p^2+q^2+2h^2=q^2+b^2=c^2=(p+q)^2=p^2+2pq+q^2 \Rightarrow 2h^2=2hq\Rightarrow h^2=pq
	\end{align*}
	
	\item
	\begin{align*}
	a^2=h^2+q^2=pq+q^2=q(p+q)=qc\\
	b^2=h^2+p^2=pq+p^2=p(p+q)=pc
	\end{align*}
	
	\end{enumerate}\todo{Skizzen, Beweis 3.?}


\subsection{Beispiel}

	Sei $\Delta ABC$ ein gleichseitiges Dreieck.\todo{Skizze}\\
	\newline
	Für die Höhe $h$ gilt dann: $h^2=\left(\frac{a}{2}\right)^2=a^2\Rightarrow h=\frac{\sqrt{3}}{2}a$.\\
	Für die Fläche folgt: $F=\frac{\sqrt{3}}{4}a^2$.


\subsection{Beispiel}

	
	Sei $\Delta ABC$ gleichschenklig-rechtwinklig.\todo{Skizze}\\
	\newline
	Es gilt: $c^2=2a^2\Rightarrow c=\sqrt{2}a$\\
	und $h^2+\left(\frac{c}{2}\right)^2=a^2\Rightarrow h^2=\frac{a^2}{2}\Rightarrow h=\frac{\sqrt{2}}{2}a$.\\
	Schließlich folgt: $F=\frac{1}{2}a^2$.
	
%\section{Trigonometrie}

%\include{sections/08-raumgeometrie}
%\include{sections/09-analytische_geometrie}
%\include{sections/10-abbildungsgeometrie}
\setcounter{section}{10}
\section{Die komplexen Zahlen}

\subsection{Definition}

	\begin{enumerate}
	\item Wir führen auf $\mathbb{R}^2$ eine Multiplikation ein durch
	\begin{align*}
	(a,b)\cdot (c,d) = (ac-bd, bc-ad).
	\end{align*}
	Wie man leicht nachprüft, erhält man damit einen Körper $\mathbb{C}$. Dieser enthält $\mathbb{R}$ mittels der 
	injektiven Abbildung
	\begin{align*}
	\iota: \mathbb{R} \hookrightarrow \mathbb{C}, a\mapsto (a,0)
	\end{align*}
	
	\item \underline{Schreibweisen:}
		\begin{itemize}
		\item Statt $e_1=(1,0)$ schreibe $1$.
		\item Statt $e_2=(0,1)$ schreibe $i$.
		\end{itemize}
		Jedes Element von $\mathbb{C}$ hat also eine eindeutige Darstellung der Form
		\begin{align*}
		a+b\cdot i \textnormal{ mit } a,b \in \mathbb{R}.
		\end{align*}
		
	\item Für $z=a+bi \in \mathbb{C}$ heißt $\operatorname{Re}(z)=a$ der \textbf{Realteil} von $z$ 
	und $\operatorname{Im}(z)=b$ der \textbf{Imaginärteil} von $z$.
	
	\end{enumerate}
	
	
\subsection{Bemerkung}

	\begin{enumerate}
	\item Die Zahl $i$ erfüllt $i^2 = 1$, denn $(0,1)\cdot (0,1)=(0-1,0)=(-1,0).$\\
	Man schreibt daher auch $i=\sqrt{-1}$.
	
	\item Für $a\in \mathbb{R}_{\geq o}$ gilt:
	\begin{align*}
	(\sqrt{a}\cdot i)^2=a\cdot (-1)=-a \textnormal{, also } \pm\sqrt{a}\cdot i = \sqrt{-a}
	\end{align*}
	
	\item Für $z=a+bi \in \mathbb{C}\backslash \{0\}$ gilt:
	\begin{align*}
	\frac{1}{z}=\frac{1}{a+bi}=\frac{a-bi}{(a+bi)(a-bi)}=\frac{a-bi}{a^2+b^2}=
	\frac{a}{a^2+b^2}-\frac{b}{a^2+b^2}\cdot i
	\end{align*}
	\end{enumerate}
	
	
\subsection{Beispiele}

	\begin{enumerate}
	\item Für $n\in\mathbb{Z}$ gilt: 
	$i^n=\left\{\begin{array}{ccc}
          		i	& \textnormal{falls}	& n\equiv 1 \quad (mod~4) \\
          		-1	& \textnormal{falls}	& n\equiv 2 \quad (mod~4) \\
          		-i	& \textnormal{falls}	& n\equiv 3 \quad (mod~4) \\
          		1	& \textnormal{falls}	& n\equiv 0 \quad (mod~4) \\                  			
          \end{array}\right.$.
	
	
	\item $(1+i)^2=2i$
	\item In $\mathbb{C}$ besitzt jede quadratische Gleichung genau 2 Lösungen.
	\end{enumerate}
	
	
\subsection[Komplexe Konjugation]{Definition}

	Die Abbildung $\mathlll{\kappa}:\mathbb{C}\rightarrow\mathbb{C},\quad a+bi\mapsto a-bi$ 
	heißt die \textbf{komplexe Konjugation}.\\ \todo{Funktion schöner machen}
	Für $z\in\mathbb{C}$ schreiben wir statt $\mathlll{\kappa}(z)$ auch $\overline{z}$.
	
	
\subsection[Rechenregeln für komplexe Konjugation]{Bemerkung (Rechenregeln für komplexe Konjugation)}

	\begin{enumerate}
	\item $\overline{\overline{z}}=z$
	\item $\overline{z_1+z_2}=\overline{z_1}+\overline{z_2}$
	\item $\overline{z_1\cdot z_2}=\overline{z_1}\cdot\overline{z_2}$
	\item Für $a\in\mathbb{R}$ gilt: $\overline{a\cdot z}= a\cdot \overline{z}$
	\item $\frac{1}{\overline{z}}=\overline{\left(\frac{1}{z}\right)}$
	\item $\overline{z}+z=2a=2\operatorname{Re}(z)$
	\item $z-\overline{z}=2bi=2\operatorname{Im}(z)\cdot i$
	\end{enumerate}\todo{7.?}
	
	
\subsection[Die komplexe Zahlenebene]{Bemerkung (Die komplexe Zahlenebene)}

	Im $\mathbb{A}^2(\mathbb{R})$ führen wir kartesische Koordinaten ein und identifizieren die Zahl
	$z = a+bi\in \mathbb{C}$ mit dem Punkt $(a,b)$.\todo{Skizze}\\ \newline
	
	\underline{Geometrische Interpretation der Körperoperationen:}\\
	\
	\begin{enumerate}
	\item Addition:
	\begin{align*}
	(a+bi)+(c+di)=(a+c)+(b+d)i \textnormal{entspricht der Vektoraddition.}
	\end{align*}
	
	\item Multiplikation mit $i: \qquad (a+bi)\cdot i=-b+ai$.\\ \newline
	Dies entspricht also der Drehung um $90^{\circ}$ im mathematisch positiven Sinn um den Ursprung.

	\item Abstand zum Nullpunkt/Länge des Vektors\\
	Der \textbf{Betrag} einer komplexen Zahl $z=a+bi$ ist $\vert z \vert=\sqrt{a^2+b^2}$\\ \newline
	\underline{Eigenschaften:}
		\begin{enumerate}
		\item $\vert z_1\cdot z_2\vert = \vert z_1\vert\cdot\vert z_2\vert$ für $z_1,z_2\in\mathbb{C}$
		\item $\vert c\cdot z\vert = \vert c\vert\cdot\vert z\vert$ für $c\in\mathbb{R},z\in\mathbb{C}$
		\item $\vert z_1 + z_2\vert \leq \vert z_1\vert + \vert z_2\vert$ für $z_1,z_2\in\mathbb{C}$
		\end{enumerate}
		
	\item Der \textbf{Winkel} $\varphi$ (oder das \textbf{Argument}) einer komplexen Zahl 
	$z\in\mathbb{C}\backslash\{0\}$ mit $z=a+bi$ und $a,b\in\mathbb{R}$ erfüllt 
	$z=\vert z\vert\cdot \cos(\varphi) + \vert z \vert \cdot \sin(\varphi)\cdot i 
	\qquad$ \mbox{$(\varphi\in[0,2\pi[)$}.\\ \newline
	Wir schreiben $\varphi=\operatorname{arc}(z).$
	
	\item Sei $r=\vert z\vert$ und $\varphi=\operatorname{arc}(z)$. Dann ist die Multiplikation mit
	\begin{align*}
	z=r\cdot\cos(\varphi)+r\cdot\sin(\varphi)i=r\big( \cos(\varphi)+\sin(\varphi)i~\big)
	\end{align*}
	die Komposition der Multiplikation mit $\cos(\varphi)+\sin(\varphi)i$ 
	und der Multiplikation mit $r\in\mathbb{R}_+$. Letztere ist die zentrische Streckung um den Faktor $r$ mit 
	Mittelpunkt 0.\\ \newline
	Nun wende die erste Multiplikation an auf $\tilde{z}=\tilde{r}\cdot (\cos(\psi)+\sin(\psi)i)$.\\
	Dann gilt:
	\begin{equation*}\begin{gathered}
	\tilde{z}\cdot\big( \cos(\varphi)+\sin(\varphi)i \big)=\\
	\tilde{r}\big( \cos(\psi)+\sin(\psi)i \big)\big( \cos(\varphi)+\sin(\varphi)i \big)=\\
	\tilde{r}\Big[\big(\cos(\psi)\cos(\varphi)-\sin(\psi)\sin(\varphi)\big)+
	\big(\sin(\psi)\cos(\varphi)+\cos(\psi)\sin(\varphi)\big)i\Big]=\\
	\tilde{r}\Big[\cos(\psi+\varphi) + \sin(\psi+\varphi)i \Big]
	\end{gathered}\end{equation*}
	Das Produkt hat also denselben Betrag, aber der Winkel ist um $\varphi$ größer. Die Multiplikation mit 
	$\cos(\varphi)+\sin(\varphi)i$ entspricht also der Drehung um den Winkel $\varphi$ um den Nullpunkt.\\
	\newline
	Insgesamt ist die Multiplikation mit $z$ also eine Drehstreckung mit Zentrum 0.
	
	\end{enumerate}\todo{Skizzen!s}


\subsection[Algebraische Beschreibung geometrischer Mengen]{Bemerkung (Algebraische Beschreibung geometrischer Mengen)}

	\begin{enumerate}
	\item (Kreis mit Mittelpunkt $m\in\mathbb{C}$ und Radius $r\in\mathbb{R}_+$)\\
	Der Abstand von $z\in\mathbb{C}$ zu $m$ ist gegeben durch $\vline z-m \vline$.\\
	Der Kreis ist also gegeben durch $K=\{ z\in\mathbb{C}~|~\vline z-r \vline = r \}$.
	
	\item Eine Gerade $G$ durch $z=a+bi$ und $\tilde{z}=c+di$ ist gegeben durch 
	$G=\{z+\lambda(\tilde{z}-z)|\lambda\in\mathbb{R} \}$ (\textit{explizite Darstellung}).\\ \newline
	\textit{Implizite Darstellung:} $G=\{z\in\mathbb{C}|\overline{e}z+e\overline{z}+g=0\}$
	\mbox{mit $e\in\mathbb{C},~g\in\mathbb{R}$}.\\
	Es gilt: $\overline{e}z+e\overline{z}=\overline{e}z+\overline{(ez)}=2\operatorname{Re}(\overline{e}z)$.\\
	Schreibe: $e=\alpha+\beta i$ mit $\alpha,\beta\in\mathbb{R}$ und $z=x+yi$ mit $x,y\in\mathbb{R}$.\\
	Dann folgt: 
	\begin{equation*}\begin{gathered}
	\operatorname{Re}(\overline{e}z)=\operatorname{Re}\big( (\alpha+\beta i)(x+yi) \big)=\\
	\operatorname{Re}\big( (\alpha x+\beta y)+(-\beta x+\alpha y)i \big)=\\
	\alpha x+\beta y.\\
	\Bigg(= \Bigg \langle 
	\left(\begin{array}{c}\alpha\\ \beta \end{array}\right),
	\left(\begin{array}{c} x\\y\end{array}\right)
	\Bigg \rangle\Bigg)
	\end{gathered}\end{equation*}
	~\newline		
	Dann folgt:
	\begin{align*}
	\overline{e}z+e\overline{z}+\underbrace{g}_{\in\mathbb{R}}=
	\underbrace{2\alpha x+2\beta y+g=0}_{\textnormal{implizite Geradengleichung in }\mathbb{R}^2.}.
	\end{align*}
	
	\end{enumerate}
	
	
\subsection[Algebraische Interpretation von Abbildungen der Zeichenebene]{Bemerkung (Algebraische Interpretation von Abbildungen der Zeichenebene)}

	\begin{enumerate}
	\item (Translation)\\
	Die Translation $\tau$ um den Vektor $a+bi$ ist die Addition
	\begin{align*}
	\tau : \mathbb{C}\rightarrow\mathbb{C}, z\mapsto z+(a+bi)
	\end{align*}.
	
	\item (Drehung)\\
	Die Drehung um den Winkel $\varphi$ um den Nullpunkt entspricht der Multiplikation mit 
	$\cos(\varphi)+\sin(\varphi)i$.\\ \newline
	Sei nun $m\in\mathbb{C}$.\\ \newline
	Die Drehung $\mathlll{\rho}{m,\varphi}$ ist die Komposition von
		\begin{enumerate}
		\item Verschiebung um $-m$
		\item Drehung um 0 um Winkel $\varphi$
		\item Verschiebung um $+m$
		\end{enumerate}
		~\newline
	Es gilt also: $\mathlll{\rho}_{m,\varphi}: \mathbb{C}\rightarrow\mathbb{C}, 
	z\mapsto (\cos(\varphi)+\sin(\varphi)i)(z-m)+m$.
	
	\item (Spiegelung an einer Geraden durch 0)\\
	$G$ habe einen Richtungsvektor $a+bi$.\\ 
	Die Gerade durch $z$ und $\tilde{z}$ haben dann den Richtungsvektor	$-b+ai$.\\
	Dann ist $G\cap H$ der Lotfußpunkt $l\in\mathbb{C}$ und es gilt:
	\begin{align*}
	\tilde{z}=\mathlll{\sigma}_G(z)=z+2(l-z)=-z+2l.
	\end{align*}
	
	\end{enumerate}\todo{Funktionen, Skizzen}
	
	
\subsection[Geometrische Interpretation von $z\mapsto\overline{z}$]{Bemerkung (Geometrische Interpretation von $z\mapsto\overline{z}$)}\todo{fett}

	Die Abbildung \todo{Funktion} $\mathlll{\kappa}: \mathbb{C}\rightarrow\mathbb{C},a+bi\mapsto a-bi$ entspricht der Spiegelung an der $\operatorname{Re}(z)$-Achse.
	
	
\subsection[Inversion am Kreis]{Bemerkung (Inversion am Kreis)}


	\todo{Funktion}Die Abbildung $\iota:\mathbb{C}\backslash\{0\}\rightarrow\mathbb{C}\backslash\{0\},
	z\mapsto \frac{1}{\overline{z}}$ wirkt auf $z=a+bi=r(\cos(\varphi)+\sin(\varphi)i)$ wie folgt:
	
	\begin{equation*}
	\begin{gathered}
	\iota(z)=\frac{1}{\overline{z}}=\\
	\mathll{\kappa}\left( \frac{1}{r}\cdot \frac{\cos(\varphi)-\sin(\varphi)i}{(\cos(\varphi)+\sin(\varphi)i)
	(\cos(\varphi)-\sin(\varphi)i)}  \right)=\\
	\frac{1}{r}\mathll{\kappa}(\cos(\varphi)-\sin(\varphi)i)=\\
	\frac{1}{r}(\cos(\varphi)+\sin(\varphi)i)	
	\end{gathered}
	\end{equation*}
	
	Die Zahl $\iota(z)$ liegt also auf demselben Halbstrahl druch 0 wie $z$. Ihr Betrag ist $\frac{1}{r}.$
	\todo{skizze}\\ \newline
	
	Der Einheitskreis $\mathbb{E}=\{\cos(\varphi)+\sin(\varphi)i|\varphi\in [0,2\pi [\}$ bleibt dabei fest.\\
	\newline
	Es gilt: $\iota^2(z)=\iota(\iota(z))=z$.\\ \newline
	Die geometrische Interpretation von $\frac{1}{z}$ ist also der Punkt, der aus $z$ entsteht, 
	indem man zuerst an der $\operatorname{Re}(x)$-Achse spiegelt und dann die Inversion am 
	Einheitskreis durchführt.
	
	
\subsection[Eigentschaften der Inversion am Kreis]{Satz (Eigentschaften der Inversion am Kreis)}

	\begin{enumerate}
	\item Ist $G$ eine Gerade in $\mathbb{C}$, die nicht durch 0 geht, so ist $\iota(G)$ ein Kreis.
	\item Ist $G$ eine Gerade durch 0, so gilt $\iota(G\backslash\{0\})=G\backslash\{0\}$.
	\item Ist $K$ ein Kreis in $\mathbb{C}$, der durch 0 geht, so ist $\iota(K\backslash\{0\})$ eine Gerade,
	die nicht durch 0 geht.
	\item Ist $K$ ein Kreis in $\mathbb{C}$, der nicht durch 0 geht, so ist $\iota(K)$ wieder ein Kreis.
	\end{enumerate}


\subsection[Die komplexe $e$-Funktion]{Bemerkung (Die komplexe $e$-Funktion)}

	Die reelle $e$-Funktion $x\mapsto e^x$ erfüllt $(e^x)'=e^x$ und $e^0=1$.\\
	Für alle $x\in\mathbb{R}$ gilt: \[
	e^x=1+\frac{1}{1!}x+\frac{1}{2!}x^2+\dots =\sum_{n\geq 0} \frac{1}{n!}x^n \].\\ \newline
	
	Für $z\in\mathbb{C}$ konvergiert die Reihe \[\sum_{n\geq 0} \frac{1}{n!}z^n \in\mathbb{C} \] ebenfalls.\\
	\newline
	Weitere Potenzreihen, die für jedes $x\in\mathbb{R}$ konvergieren, sind:
	
	\begin{itemize}
	\item $\sin x=\frac{1}{1!}x-\frac{1}{3!}x^3+\frac{1}{5!}x^5-+\dots$
	\item $\cos x=1-\frac{1}{2!}x^2+\frac{1}{4!}x^4-\frac{1}{6!}x^6+-\dots$
	\end{itemize}

	~\newline
	Durch dieselben Potenzreihen erhält man $e^z,\sin(z)$ und $\cos(z)$ auch für alle $z\in\mathbb{C}$.\\ \newline
	Man weißt nach, dass $e^{z_1+z_2}=e^{z_1}\cdot e^{z_2}$ für alle $z_1,z_2 \in\mathbb{C}$ gilt.\\ \newline 
	
	
\subsection[Die Eulersche Formel]{Satz (Die Eulersche Formel)}

	Für alle $\varphi\in\mathbb{R}$ gilt: 
	$e^{i\varphi}=\underbrace{\cos(\varphi)+i\cdot\cos(\varphi)}_{\textnormal{Punkt auf dem Einheitskreis}}$\\
	\newline
	\textsc{Beweis.} Es gilt:
	\begin{align*}
	e^{i\varphi} 	&=1+\frac{1}{1!}(i\varphi)+\frac{1}{2!}(i^2\varphi^2)+\frac{1}{3!}(i^3\varphi^3)+\dots=\\
					&=(1-\frac{1}{2!}\varphi^2+\frac{1}{4!}\varphi^4-\frac{1}{6!}\varphi^6+-\dots)+
					i(\frac{1}{1!}\varphi-\frac{1}{3!}\varphi^3+\frac{1}{5!}\varphi^5-+\dots)=\\
					&=\cos\varphi+i\cdot\sin\varphi 
	\end{align*}\todo{qed}
	
	
\subsection{Korollar}

	\begin{align*}
	e^{i\pi}+1=0.
	\end{align*}


\subsection{Beispiel}

	Aus der Eulerschen Formel folgt:
	\begin{align*}
	e^{i(\alpha+\beta)}=\cos(\alpha+\beta)+i\cdot\sin(\alpha\beta).
	\end{align*}
	Ferner gilt:
	\begin{align*}
	e^{i(\alpha+\beta)}	&=e^{i\alpha}\cdot e^{i\beta}=\\
						&=(\cos(\alpha)+i\cdot \sin(\alpha))(\cos(\beta)+i\sin(\beta))=\\
						&=[\cos\alpha\cdot\cos\beta-\sin\alpha\sin\beta]+
						i[\sin\alpha\cos\beta+\cos\alpha\sin\beta].
	\end{align*}
	
	Durch Vergleich der Real- und Imaginärteile folgen die Additionstheoreme für $\sin$ und $\cos$.

\section{Kombinatorik}

Kombinatorik ist die Kunst des Zählens.


\subsection{Definition}

	Seien $a_1,\dots,a_n$ paarweise verschiedene Objekte ($n\geq 1$).
	
	\begin{enumerate}
	\item Eine Anordnung $(a_{i_1},a_{i_2},\dots,a_{i_n})$ mit $\{ i_i,\dots, i_n \}=\{ 1,\dots,n \}$ heißt auch
	\textbf{Permutation} von $a_1,\dots,a_n$.\\ \newline
	\underline{Schreibweisen:}\\ \newline 
	$\sigma=\left(\begin{array}{cccc}
   				a_1		& a_2		& \dots & a_n\\
        		a_{i_1}	& a_{i_2}	& \dots & a_{i_n}
            \end{array}\right)$ oder einfach
   	$\sigma=\left(\begin{array}{cccc}
                1	& 2		& \dots & n\\
                i_1	& i_2	& \dots & i_n
            \end{array}\right)$.\\ \newline
	Ohne Einschränkung betrachten wir also meißt die Permutationen der Menge $\{1,\dots,n\}$.
	\item Die Menge aller Permutationen von $n$ Objekten heißt die \textbf{symmetrische Gruppe $\mathbf{S_n}$}.
	\end{enumerate}
	
	
\subsection{Satz: Die Gruppe $S_n$ hat $n!$ Elemente.}

	\textsc{Beweis.} Halte ein Element, zB $a_n$ fest. Für die Bilder $\sigma(a_1)$ unter $\sigma\in S_n$ gibt
	es $n$ Möglichkeiten, für $\sigma(a_2)$ gibt es dann noch $n-1$ Möglichkeiten usw.\\
	Am Ende gibt es für $\sigma(a_n)$ nur noch 1 Auswahl.\
	Insgesamt gibt es $n\cdot (n-1)\cdot (n-2)\cdot \ldots \cdot 2 \cdot 1 = n!$ Permutationen. \todo{qed}
	
	
\subsection[Beispiel]{Beispiel: Wir ordnen Permutationen}

	\textit{Dieser Punkt wurde während der Ausführung gestrichen, 
	da der Kentnissstand der Studierenden in Lineare Algebra nicht ausreichend war.}
	
	
\subsection{Definition}

	Gegeben seien $n$ paarweise verschiedene Objekte $a_1,\dots,a_n$.
	
	\begin{enumerate}
	\item Sei $0\leq m \leq n$. Eine Teilmenge von $\{a_1,\dots,a_n\}$ bestehend aus $m$ Elementen heißt auch
	\textbf{Auswahl} von $m$ Elementen.
	
	\item Die Anzahl der Auswahlen von $m$ Elementen aus $\{a_1,\dots,a_n\}$ heißt der 
	\textbf{Binomialkoeffizient} ${n \choose m}$.
	
	\end{enumerate}


\subsection[Formel für die Binomialkoeffizienten]{Satz (Formel für die Binomialkoeffizienten)}

	Für $n\geq 1$ und $0\leq m \leq n$ gilt: 
	${n \choose m}=\frac{n(n-1)(n-2)\ldots (n-(m+n))}{1\cdot 2\cdot \ldots \cdot m}=\frac{n!}{m!(n-m)!}$
	(mit $0!=1$).
	
	
\subsection[Das Pascalsche Dreieck]{Bemerkung (Das Pascalsche Dreieck)}

	\begin{enumerate}
	\item Die Binomialkoeffizienten erfüllen die Formel
	\begin{align*}
	{n \choose m}={n-1 \choose m}+{n-1 \choose m-1} \textnormal{ für } m\geq 1, n\geq 2.
	\end{align*}
	
	\item Die Biomialkoeffizienten sind gegeben durch das \textbf{Pascalsche Dreieck}:\\
		
		\begin{tabular}{lcccccccccccccl}
		n=1 &  &   &   &   &   & 1  &   & 1  &   &   &   &   &  & (m=0,1)     \\
		n=2 &  &   &   &   & 1 &    & 2 &    & 1 &   &   &   &  & (m=0,1,2)   \\
		n=3 &  &   &   & 1 &   & 3  &   & 3  &   & 1 &   &   &  & (m=0,...,3) \\
		n=4 &  &   & 1 &   & 4 &    & 6 &    & 4 &   & 1 &   &  & (m=0,...,4) \\
		n=5 &  & 1 &   & 5 &   & 10 &   & 10 &   & 5 &   & 1 &  & (m=0,...,5) \\
			&  &   &   &   &   &	&\textit{etc.}&	 &	 &	 &	 &	 &	&
		\end{tabular}
		
	\end{enumerate}




\end{document}
