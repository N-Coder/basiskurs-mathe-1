\section{Kombinatorik}

Kombinatorik ist die Kunst des Zählens.


\subsection{Definition}

	Seien $a_1,\dots,a_n$ paarweise verschiedene Objekte ($n\geq 1$).
	
	\begin{enumerate}
	\item Eine Anordnung $(a_{i_1},a_{i_2},\dots,a_{i_n})$ mit $\{ i_i,\dots, i_n \}=\{ 1,\dots,n \}$ heißt auch
	\textbf{Permutation} von $a_1,\dots,a_n$.\\ \newline
	\underline{Schreibweisen:}\\ \newline 
	$\sigma=\left(\begin{array}{cccc}
   				a_1		& a_2		& \dots & a_n\\
        		a_{i_1}	& a_{i_2}	& \dots & a_{i_n}
            \end{array}\right)$ oder einfach
   	$\sigma=\left(\begin{array}{cccc}
                1	& 2		& \dots & n\\
                i_1	& i_2	& \dots & i_n
            \end{array}\right)$.\\ \newline
	Ohne Einschränkung betrachten wir also meißt die Permutationen der Menge $\{1,\dots,n\}$.
	\item Die Menge aller Permutationen von $n$ Objekten heißt die \textbf{symmetrische Gruppe $\mathbf{S_n}$}.
	\end{enumerate}
	
	
\subsection{Satz: Die Gruppe $S_n$ hat $n!$ Elemente.}

	\textsc{Beweis.} Halte ein Element, zB $a_n$ fest. Für die Bilder $\sigma(a_1)$ unter $\sigma\in S_n$ gibt
	es $n$ Möglichkeiten, für $\sigma(a_2)$ gibt es dann noch $n-1$ Möglichkeiten usw.\\
	Am Ende gibt es für $\sigma(a_n)$ nur noch 1 Auswahl.\
	Insgesamt gibt es $n\cdot (n-1)\cdot (n-2)\cdot \ldots \cdot 2 \cdot 1 = n!$ Permutationen. \todo{qed}
	
	
\subsection[Beispiel]{Beispiel: Wir ordnen Permutationen}

	\textit{Dieser Punkt wurde während der Ausführung gestrichen, 
	da der Kentnissstand der Studierenden in Lineare Algebra nicht ausreichend war.}
	
	
\subsection{Definition}

	Gegeben seien $n$ paarweise verschiedene Objekte $a_1,\dots,a_n$.
	
	\begin{enumerate}
	\item Sei $0\leq m \leq n$. Eine Teilmenge von $\{a_1,\dots,a_n\}$ bestehend aus $m$ Elementen heißt auch
	\textbf{Auswahl} von $m$ Elementen.
	
	\item Die Anzahl der Auswahlen von $m$ Elementen aus $\{a_1,\dots,a_n\}$ heißt der 
	\textbf{Binomialkoeffizient} ${n \choose m}$.
	
	\end{enumerate}


\subsection[Formel für die Binomialkoeffizienten]{Satz (Formel für die Binomialkoeffizienten)}

	Für $n\geq 1$ und $0\leq m \leq n$ gilt: 
	${n \choose m}=\frac{n(n-1)(n-2)\ldots (n-(m+n))}{1\cdot 2\cdot \ldots \cdot m}=\frac{n!}{m!(n-m)!}$
	(mit $0!=1$).
	
	
\subsection[Das Pascalsche Dreieck]{Bemerkung (Das Pascalsche Dreieck)}

	\begin{enumerate}
	\item Die Binomialkoeffizienten erfüllen die Formel
	\begin{align*}
	{n \choose m}={n-1 \choose m}+{n-1 \choose m-1} \textnormal{ für } m\geq 1, n\geq 2.
	\end{align*}
	
	\item Die Biomialkoeffizienten sind gegeben durch das \textbf{Pascalsche Dreieck}:\\
		
		\begin{tabular}{lcccccccccccccl}
		n=1 &  &   &   &   &   & 1  &   & 1  &   &   &   &   &  & (m=0,1)     \\
		n=2 &  &   &   &   & 1 &    & 2 &    & 1 &   &   &   &  & (m=0,1,2)   \\
		n=3 &  &   &   & 1 &   & 3  &   & 3  &   & 1 &   &   &  & (m=0,...,3) \\
		n=4 &  &   & 1 &   & 4 &    & 6 &    & 4 &   & 1 &   &  & (m=0,...,4) \\
		n=5 &  & 1 &   & 5 &   & 10 &   & 10 &   & 5 &   & 1 &  & (m=0,...,5) \\
			&  &   &   &   &   &	&\textit{etc.}&	 &	 &	 &	 &	 &	&
		\end{tabular}
		
	\end{enumerate}
