\section{Rechnen mit ganzen Zahlen}
$\mathbb{N} = \{0, 1, 2, 3,  \dots \}$ Menge der \textbf{natürlichen Zahlen} \newline
$\mathbb{Z} = \{0, 1, -1, 2, -2,  \dots \}$ Menge der \textbf{ganzen Zahlen}


\subsection[Zahlensysteme]{Satz (Zahlensysteme)}

	Sei $b \in \mathbb{N} \textnormal{ mit } b \geq 2 $. 
	(Die Zahl $b$ heißt \textbf{Basis} des Zahlensystems)\newline
	Dann gibt es zu jeder natürlichen Zahl $n \in \mathbb{N}$ eindeutig bestimmte Elemente 
	$a_0, a_1, \dots,a_k \in \{0,1, \dots, b-1 \} $ sodass gilt:

	\begin{align*}
		n = a_0 + a_1 \cdot b + a_2 \cdot b^2 + \cdots + a_k \cdot b^k.
	\end{align*}

	Die Zahlen $a_0,\dots,a_k$ heißen \textbf{Ziffern} von $n$ in der Darstellung zur Basis $b$. \newline
	\underline{Schreibweise:} $n_{[b]} = a_k a_{k-1} \dots a_1 a_0 $ (fehlt $[b]$ so ist $[10]$ gemeint)


\subsection{Beispiele}

	\begin{itemize}
	
		\item Binärsystem, $b=2$ \newline
		$5_{[10]}=101_{[2]}$ \newline
		$101_{[10]}=64_{[10]}+32_{[10]}+4_{[10]}+1_{[10]}=1100101_{[2]}$
		
		\item Hexadezimalsystem, $b=16$ \newline
		Notation: $10_{[10]}=A_{[16]}, 11_{[10]}=B_{[16]},\dots,15_{[10]}=F_{[16]}$ \newline
		$101_{[10]}=5\cdot 16+5=55_{[16]}$ \newline
		$1B3_{[16]}=256_{[10]}+11_{[10]}\cdot 16_{[10]}+3_{[10]}=435_{[10]}$
	
	\end{itemize}
	

\subsection[Division mit Rest]{Satz (Division mit Rest)}

	Sei $n \in \mathbb{Z}$ und $m\in \mathbb{N_+}$. \\
	Dann gibt es eine eindeutige Darstellung $n=q\cdot m+r$ mit $q\in \mathbb{Z}$ (genannt \textbf{Quotient})
	und $r \in \{0,1,\dots,m-1\}$ (genannt \textbf{Rest}).
	
	\begin{align*}
	\textnormal{\underline{Schreibweise:} } n &\equalexpl{"ist kongruent"} r \qquad (mod \ m)
	\end{align*}

	
\subsection{Beispiele}
	
	\begin{itemize}

		\item Die Zahl $n=87$ soll durch $m=5$ geteilt werden: \\
		\begin{align*}
		n = q\cdot m+r=17\cdot 5+2
		\end{align*}			
		
		\item Die möglichen Reste bei der Division einer Quadratzahl durch 12 sind: \\
		
		\centering
		\begin{tabular}{r|c|c|c|c|c|c|c|c|c|c|c|c|}
		$n$ \ (mod 12)  & 0 & 1 & 2 & 3 & 4 & 5 & 6 & 7 & 8 & 9 & 10 & 11 \\ \hline
		$n^2$ \ (mod 12) & 0 & 1 & 4 & 9 & 4 & 1 & 0 & 1 & 4 & 9 & 4  & 1
		\end{tabular}
	
	\end{itemize}

	
\subsection[Vielfaches, Teiler, Primzahl]{Definition (Vielfaches, Teiler, Primzahl)}

	\begin{enumerate}
		
		\item Ist der Rest bei der Division von $n$ durch $m$ gleich Null, 
		so heißt $n$ ein \textbf{Vielfaches} von $m$ und $m$ ein \textbf{Teiler} von $n$.
		
		\item Eine Zahl $n \geq 2$ heißt eine \textbf{Primzahl}, wenn sie nur zwei positive Teiler 1
		 und $n$ besitzt.
		
	\end{enumerate}


\subsection{Beispiele}

	\begin{itemize}
		
		\item Die Teiler von 12 sind 1, 2, 3, 4, 6, 12.
		\item Die ersten Primzahlen sind 2, 3, 5, 7, 11, 13, 17, 19, \dots
	\end{itemize}
	
	
\subsection[Fundamentalsatz der Arithmetik]{Satz (Fundamentalsatz der Arithmetik)}

	Sei $n \in \mathbb{N_+}$. Dann gibt es eine (bis auf die Reihenfolge) eindeutige Darstellung
	
	\begin{align*}
	n = p_1^{\alpha_1} \cdot p_2^{\alpha_2} \cdot \ldots \cdot p_n^{\alpha_n}
	\end{align*}
	
	mit paarweise verschiedenen Primzahlen $p_1, \dots , p_k$ und $\alpha_i \in \mathbb{N_+}$. \\
	Diese Darstellung heißt \textbf{Primfaktorzerlegung} von $n$. 

	
\subsection{Beispiele}

	\begin{itemize}
	
	\item $24 = 2^3 \cdot 3$
	\item $111 = 3 \cdot 37$
	\item $1011=7\cdot 11\cdot 13$
	\item $1024 = 2^10$
	\item $729 = 3^6$
	\item $625 = 5^4$
	
	\end{itemize}
	
	
\subsection[ggT, kgV]{Definition (ggT, kgV)}
	
	Seien $a,b \in \mathbb{N_+}$.
	
	\begin{enumerate}
	
	\item Die größte positive ganze Zahl $g \in \mathbb{N_+}$ mit $g|a$ und $g|b$ heißt der 
	\textbf{größte gemeinsame Teiler (ggT)} von $a$ und $b$.
		
	\item Die kleinste positive ganze Zahl $k \in \mathbb{N_+}$ mit $a|k$ und $b|k$ heißt das 
	\textbf{kleinste gemeinsame Vielfache (kgV)} von $a$ und $b$.
		
	\end{enumerate}


\subsection[ggT/kgV durch Primfaktorenzerlegung]{Satz (ggT/kgV durch Primfaktorenzerlegung)}

	Sei $a,b \in \mathbb{N_+}$ mit Primfaktorzerlegungen $a=p_1^{\alpha_1} \cdot \ldots \cdot p_k^{\alpha_k}$ und
	$b=p_1^{\beta_1} \cdot \ldots \cdot p_k^{\beta_k}$ mit $\alpha_i, \beta_i \geq 0$.
	Dann gilt:
	\begin{enumerate}
		
	\item $ggT(a,b)=p_1^{\gamma_1}\cdot p_2^{\gamma_2} \cdot \ldots \cdot p_k^{\gamma_k}$ 
	mit $\gamma_i=min\{\alpha_i,\beta_i \}$
		
	\item $kgV(a,b)=p_1^{\delta1}\cdot p_2^{\delta2} \cdot \ldots \cdot p_k^{\delta_k}$ 
	mit $\delta_i=max\{\alpha_i,\beta_i \}$
		
	\end{enumerate}
	
		
\subsection{Beispiele}

	\begin{itemize}
			
	\item $ggT(30,75)= 2^0\cdot 3^1\cdot 5^1 = 15$,\\ denn $30=2 \cdot 3 \cdot 5$ und $75=3 \cdot 5^2$
	\item $ggT(64,81)= 1$,\\ denn $64=2^6, 81=3^4$

	\end{itemize}	

	
\subsection[Teilbarkeitsregeln]{Bemerkung (Teilbarkeitsregeln)}	

	\begin{enumerate}
	
	\item $2|n$ genau dann, wenn die Endziffer von $n$ in $\{0,2,4,6,8\} $ ist.
	\item $3|n$ genau dann, wenn die Quersumme($Qs$) von $n$ durch 3 Teilbar ist.
	\item $4|n$ genau dann, wenn $4|(10a_1+a_0)$.
	\item $5|n$ genau dann, wenn $a_0 \in \{0,5\}$ gilt.
	\item $6|n$ genau dann, wenn $2|n$ und $3|n$.
	\item $8|n$ genau dann, wenn $8|(100a_2+10a_1+a_0)$.
	\item $9|n$ genau dann, wenn $9|Qs(n)$.
	\item $10|n$ genau dann, wenn $a_0=0$ gilt.
	\item $11|n$ genau dann, wenn $11|(a_0-a_1+a_2-+\cdots \pm a_k)$.
	\item $12|n$ genau dann, wenn $3|n$ und $4|n$.
	
	\end{enumerate}


\subsection{Beispiele}

	 \begin{itemize}
	 
	 \item $9|123453$
	 \item $11|1232$
	 
	 \end{itemize}


\subsection[Geschicktes Rechnen]{Bemerkung (Geschicktes Rechnen)}

	\begin{enumerate}
	
	\item Dritte binomische Formel: $(x-y)(x+y)=x^2-y^2$ plus Quadratzahlen
		\begin{itemize}
		\item $13 \cdot 17 = 15^2 -2^2 = 225-4=221$
		\item $23 \cdot 25 = 576-1=575$
		\item $27 \cdot  33 = 900-9=891$ 
		\end{itemize}		
		
	\item Multiplikation durch Umsortierung der Primfaktoren
		\begin{itemize}
		\item $8 \cdot 375=8 \cdot 3 \cdot 125 = 10^3 \cdot 3=3000$
		\item $40 \cdot 75=4 \cdot 10 \cdot 3 \cdot 25 = 3000$
		\end{itemize}
		
	\item Quadrieren mittels erster binomischer Formel: $(x+y)^2=x^2+2xy+y^2$
		\begin{itemize}
		\item $43^2=40^2+2 \cdot 3 \cdot 40 + 9 = 1600+240+9=1849$
		\item $98^2 \cdot (100-2)^2=10000-400+4=9604$
		\end{itemize}

	\end{enumerate}


\subsection[Rekursive Definition von ggT und kgV]{Definition (Rekursive Definition von ggT und kgV)}

	Für $n\geq 2$ und $a_0,\dots,a_n \in \mathbb{N_+}$ gilt:
	\begin{itemize}
	\item $ggT(a_1,a_2,\dots , a_n)=ggT(ggT(a_1,a_2,\dots , a_{n-1}),a_n)$
	\item $kgV(a_1,a_2,\dots , a_n)=kgV(kgV(a_1,a_2,\dots , a_{n-1}),a_n)$
	\end{itemize}
	
\subsection[Unendlichkeitssatz der Primzahlen]{Satz: Es gibt unendlich viele Primzahlen}

	\textsc{Beweis.} Angenommen es gibt nur endlich viele Primzahlen $p_1,p_2,\dots,p_k$. 
	Dann betrachte die Primfaktorenzerlegung von $n=p_1 \cdot p_2 \cdot \ldots \cdot p_k +1$. 
	Die Zahlen $p_1,p_2,\dots,p_k$ teilen $n$ \underline{nicht}, sondern lassen den Rest 1. 
	Also sind $p_1,p_2,\dots,p_k$ nicht alle Primzahlen.\todo{blitz, qed}
