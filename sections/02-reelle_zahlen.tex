\section{Rechnen mit Brüchen und Reellen Zahlen}

$\mathbb{Q} = \{ \frac{a}{b}| a \in \mathbb{Z}, b \in \mathbb{N_+} \}$ Menge der \textbf{rationalen Zahlen}

\subsection[Rechenregeln für Brüche]{Bemerkung (Rechenregeln für Brüche)}

	Für alle $a,c \in \mathbb{Z}$ und $b,c\in \mathbb{N_+}$ gilt:
	
	\begin{enumerate}
	
	\item (Gleichheit von Brüchen)\\
		\begin{align*}
			\frac{a}{b} = \frac{c}{d} \textnormal{ genau dann wenn } ad=bc
		\end{align*}
		
		Beispiel: $\frac{3}{6}=\frac{1}{2}$ \newline
		
		Kürzen von Brüchen:
		\begin{align*}
			\frac{a\cdot n}{b\cdot n} = \frac{a}{b} \textnormal{ für alle } n \in \mathbb{N_+}
		\end{align*}
		
		
	\item (Addition/Subtraktion von Brüchen)
		\begin{align*}
			\frac{a}{b} + \frac{c}{d} = \frac{ad+bc}{bd} = \frac{a\cdot\tilde{b}+c\cdot \tilde{d}}{kgV(b,d)}
		\end{align*}
		mit $\tilde{b}=\frac{kgV(b,d)}{b}$ und $\tilde{d}=\frac{kgV(b,d)}{d}$. \newline
		
		Beispiele: $\frac{1}{4}+\frac{1}{4}=\frac{2}{4}=\frac{1}{2}, 
		\frac{7}{30}+\frac{11}{45}=\frac{22}{90}+\frac{22}{90}=\frac{43}{90}$
		
	\item (Multiplikation von Brüchen)
		\begin{align*}
			\frac{a}{b} \cdot \frac{c}{d} = \frac{a\cdot c}{b\cdot d}
		\end{align*}			
	
	\item (Division von Brüchen/Doppelbrüche)
		Sei nun $c\neq 0$. Dann gilt:
		
		\begin{align*}
			\frac{\frac{a}{b}}{\frac{c}{d}}= \frac{a}{b}\cdot \frac{d}{c}=\frac{ad}{bc}
		\end{align*}
		
	\item (Kehrwert eines Bruchs)
		\begin{align*}
		\left( \frac{a}{b} \right) ^{-1} = \frac{1}{\frac{a}{b}}=\frac{b}{a}
			\textnormal{ falls } a \in \mathbb{Z}\backslash\{0\}
		\end{align*}
			
	\end{enumerate}
	

\subsection{Beispiele}

	 \begin{enumerate}
	 
	 \item Für $n \geq 1$ gilt
	 \begin{align*}
	 	\frac{1}{m} - \frac{1}{m-1} = \frac{m+1}{m(m+1)}-\frac{m}{m(m+1)} = \frac{1}{m(m+1)},
	 \end{align*}
	  
	  also zB $\frac{1}{3}-\frac{1}{4}=\frac{1}{12}$.
	  
	  \item \begin{itemize}
	  		\item $\frac{1}{2}+\frac{1}{4}=\frac{3}{4}$
	  		\item $\frac{1}{2}+\frac{1}{4}+\frac{1}{8}=\frac{7}{8}$
	  		\item $\frac{1}{2} +\frac{1}{4}+\frac{1}{8}+\frac{1}{16}=\frac{15}{16}$
	  		\item $\frac{1}{2} +\frac{1}{4}+\frac{1}{8}+\dots+\frac{1}{2^n}=\frac{2^n-1}{2^n}$
	  		\end{itemize}
	 
	 \end{enumerate}
	 
	 
\subsection[Potenzen]{Definition (Potenzen)}

	\begin{enumerate}
	\item Sei $a \in \mathbb{R}$. Dann definiere $a^0=1, a^1=a, a^2=a^1\cdot a = a\cdot a$ etc.\\
	Für $n\geq 1$ sei also $a^n= a^{n-1}\cdot a = \underbrace{a \cdot a \cdot \ldots \cdot a}_{n-mal}$.\\
	Die Zahl $a^n$ heißt die $n$-te Potenz von $a$.
	
	\item Sei $a \in \mathbb{R}$ mit $a \neq 0$. Für $n=-k$ mit $k\geq 1$ setze $a^n = a^{-k}= \frac{1}{a^k}$.\\
	
	\end{enumerate}


\subsection{Beispiele}
	
	\begin{itemize}
	\item $343 = 7^3§$
	\item $2^{-3}=\frac{1}{2^3}=\frac{1}{8}=0,125$
	\item $a^{-2}=\frac{1}{a^2}$
	\item $3^6=9^3=729$
	\end{itemize}
	
	
\subsection[Rechenregeln für Potenzen]{Bemerkung (Rechenregeln für Potenzen)}

	Für $a,b \in \mathbb{R}$ und $k,l\in \mathbb{Z}$ gilt:
	
	\begin{enumerate}
	\item $a^k \cdot b^k = (ab)^k$
	\item $a^k \cdot a^l = a^{k+l}$
	\item ${\left(  a^k \right)}^l=a^{kl}$
	\item $\frac{a^k}{a^l}=a^{k-l} \textnormal{ falls } a \neq 0$
	\item ${\left( \frac{a}{b} \right) }^k = \frac{a^k}{b^k} \textnormal{ falls } b \neq 0$
	\end{enumerate}
	
	
\subsection[Wurzeln]{Definition (Wurzeln)}

	\begin{enumerate}
	\item Sei $a \in \mathbb{R_+}=\{a \in \mathbb{R}|a > 0 \}$ und $k\in \mathbb{N_+}$.\\
	Dann gibt es genau ein $b\in \mathbb{R_+}$ mit $b^k=a$. Diese Zahl $b$ heißt die $k$-te Wurzel von $a$ 
	und wird mit $b=\sqrt[k]{a}$ bezeichnet.\\
	Im Fall $k=2$ schreiben wir auch einfach $b=\sqrt{a}$. ("Quadratwurzel") \todo{Gänsefüßchen}
	
	\item Für $a\in \mathbb{R_+}$ und $m,n\in \mathbb{N_+}$ setzen wir $a^{\frac{m}{n}}=\sqrt[n]{a^m}$.
	Insbesondere sei also $a^{\frac{1}{m}}=\sqrt[m]{a}$.\\
	Mit dieser Definition gelten die Rechenregeln für Potenzen auch für rationale Exponenten. Insbesondere
	sei $a^{-\frac{m}{n}}=\frac{1}{a^{\frac{m}{n}}}$.
	\end{enumerate}
	
	
\subsection{Beispiele}

	\begin{itemize}
	\item $\sqrt[3]{24}=\sqrt{2^3\cdot 3}=\sqrt[3]{2^3}\cdot \sqrt[3]{3}=2\cdot \sqrt[3]{3}$
	\item $\sqrt[3]{216}=6$	
	\item $\sqrt{484}=22$
	\item $\sqrt{\frac{36}{121}}=\frac{6}{11}$
	\item $\sqrt{6}\cdot \sqrt{3}= \sqrt{2}\cdot\sqrt{3}\cdot\sqrt{3}=3\sqrt{2}$	
	
	\end{itemize}
	
	
\subsection[Irrationalitätsbeweis von $\sqrt{2}$ nach Euklid]{Satz (Euklid)}
\todo{Blitz, qed}
	\begin{tabular}{ll}
	
	
	\textsc{Behauptung.} 	& $\sqrt{2}$ ist keine rationale Zahl.\\
							&									\\
	\textsc{Beweis.} 		& Angenommen $\sqrt{2}$ wäre rational.\\
							& Dann gäbe es $a,b \in \mathbb{N_+}$ mit $\sqrt{2}=\frac{a}{b}$. \\
							& Durch Kürzen können wir annehmen, dass $ggT(a,b)=1$ gilt. \\
							& Durch Quaddrieren folgt $2= \frac{a^2}{b^2}$, also $2b^2=a^2$. \\
							& Da $a^2$ gerade ist, muss auch $a$ gerade sein, 
							das heißt $\exists c \in \mathbb{N_+}$ mit $a=2c$. \\
							& Einsetzen liefert $2b^2=(2c)^2 \Leftrightarrow b^2=2c^2$. \\
							& Somit muss auch $b$ gerade sein. BLITZ zu $ggT(a,b)=1$.
	\end{tabular}